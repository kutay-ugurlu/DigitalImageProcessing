\documentclass[11pt]{article}

    \usepackage[breakable]{tcolorbox}
    \usepackage{parskip} % Stop auto-indenting (to mimic markdown behaviour)
    

    % Basic figure setup, for now with no caption control since it's done
    % automatically by Pandoc (which extracts ![](path) syntax from Markdown).
    \usepackage{graphicx}
    % Maintain compatibility with old templates. Remove in nbconvert 6.0
    \let\Oldincludegraphics\includegraphics
    % Ensure that by default, figures have no caption (until we provide a
    % proper Figure object with a Caption API and a way to capture that
    % in the conversion process - todo).
    \usepackage{caption}
    \DeclareCaptionFormat{nocaption}{}
    \captionsetup{format=nocaption,aboveskip=0pt,belowskip=0pt}

    \usepackage{float}
    \floatplacement{figure}{H} % forces figures to be placed at the correct location
    \usepackage{xcolor} % Allow colors to be defined
    \usepackage{enumerate} % Needed for markdown enumerations to work
    \usepackage{geometry} % Used to adjust the document margins
    \usepackage{amsmath} % Equations
    \usepackage{amssymb} % Equations
    \usepackage{textcomp} % defines textquotesingle
    % Hack from http://tex.stackexchange.com/a/47451/13684:
    \AtBeginDocument{%
        \def\PYZsq{\textquotesingle}% Upright quotes in Pygmentized code
    }
    \usepackage{upquote} % Upright quotes for verbatim code
    \usepackage{eurosym} % defines \euro

    \usepackage{iftex}
    \ifPDFTeX
        \usepackage[T1]{fontenc}
        \IfFileExists{alphabeta.sty}{
              \usepackage{alphabeta}
          }{
              \usepackage[mathletters]{ucs}
              \usepackage[utf8x]{inputenc}
          }
    \else
        \usepackage{fontspec}
        \usepackage{unicode-math}
    \fi

    \usepackage{fancyvrb} % verbatim replacement that allows latex
    \usepackage{grffile} % extends the file name processing of package graphics
                         % to support a larger range
    \makeatletter % fix for old versions of grffile with XeLaTeX
    \@ifpackagelater{grffile}{2019/11/01}
    {
      % Do nothing on new versions
    }
    {
      \def\Gread@@xetex#1{%
        \IfFileExists{"\Gin@base".bb}%
        {\Gread@eps{\Gin@base.bb}}%
        {\Gread@@xetex@aux#1}%
      }
    }
    \makeatother
    \usepackage[Export]{adjustbox} % Used to constrain images to a maximum size
    \adjustboxset{max size={0.9\linewidth}{0.9\paperheight}}

    % The hyperref package gives us a pdf with properly built
    % internal navigation ('pdf bookmarks' for the table of contents,
    % internal cross-reference links, web links for URLs, etc.)
    \usepackage{hyperref}
    % The default LaTeX title has an obnoxious amount of whitespace. By default,
    % titling removes some of it. It also provides customization options.
    \usepackage{titling}
    \usepackage{longtable} % longtable support required by pandoc >1.10
    \usepackage{booktabs}  % table support for pandoc > 1.12.2
    \usepackage{array}     % table support for pandoc >= 2.11.3
    \usepackage{calc}      % table minipage width calculation for pandoc >= 2.11.1
    \usepackage[inline]{enumitem} % IRkernel/repr support (it uses the enumerate* environment)
    \usepackage[normalem]{ulem} % ulem is needed to support strikethroughs (\sout)
                                % normalem makes italics be italics, not underlines
    \usepackage{mathrsfs}
    

    
    % Colors for the hyperref package
    \definecolor{urlcolor}{rgb}{0,.145,.698}
    \definecolor{linkcolor}{rgb}{.71,0.21,0.01}
    \definecolor{citecolor}{rgb}{.12,.54,.11}

    % ANSI colors
    \definecolor{ansi-black}{HTML}{3E424D}
    \definecolor{ansi-black-intense}{HTML}{282C36}
    \definecolor{ansi-red}{HTML}{E75C58}
    \definecolor{ansi-red-intense}{HTML}{B22B31}
    \definecolor{ansi-green}{HTML}{00A250}
    \definecolor{ansi-green-intense}{HTML}{007427}
    \definecolor{ansi-yellow}{HTML}{DDB62B}
    \definecolor{ansi-yellow-intense}{HTML}{B27D12}
    \definecolor{ansi-blue}{HTML}{208FFB}
    \definecolor{ansi-blue-intense}{HTML}{0065CA}
    \definecolor{ansi-magenta}{HTML}{D160C4}
    \definecolor{ansi-magenta-intense}{HTML}{A03196}
    \definecolor{ansi-cyan}{HTML}{60C6C8}
    \definecolor{ansi-cyan-intense}{HTML}{258F8F}
    \definecolor{ansi-white}{HTML}{C5C1B4}
    \definecolor{ansi-white-intense}{HTML}{A1A6B2}
    \definecolor{ansi-default-inverse-fg}{HTML}{FFFFFF}
    \definecolor{ansi-default-inverse-bg}{HTML}{000000}

    % common color for the border for error outputs.
    \definecolor{outerrorbackground}{HTML}{FFDFDF}

    % commands and environments needed by pandoc snippets
    % extracted from the output of `pandoc -s`
    \providecommand{\tightlist}{%
      \setlength{\itemsep}{0pt}\setlength{\parskip}{0pt}}
    \DefineVerbatimEnvironment{Highlighting}{Verbatim}{commandchars=\\\{\}}
    % Add ',fontsize=\small' for more characters per line
    \newenvironment{Shaded}{}{}
    \newcommand{\KeywordTok}[1]{\textcolor[rgb]{0.00,0.44,0.13}{\textbf{{#1}}}}
    \newcommand{\DataTypeTok}[1]{\textcolor[rgb]{0.56,0.13,0.00}{{#1}}}
    \newcommand{\DecValTok}[1]{\textcolor[rgb]{0.25,0.63,0.44}{{#1}}}
    \newcommand{\BaseNTok}[1]{\textcolor[rgb]{0.25,0.63,0.44}{{#1}}}
    \newcommand{\FloatTok}[1]{\textcolor[rgb]{0.25,0.63,0.44}{{#1}}}
    \newcommand{\CharTok}[1]{\textcolor[rgb]{0.25,0.44,0.63}{{#1}}}
    \newcommand{\StringTok}[1]{\textcolor[rgb]{0.25,0.44,0.63}{{#1}}}
    \newcommand{\CommentTok}[1]{\textcolor[rgb]{0.38,0.63,0.69}{\textit{{#1}}}}
    \newcommand{\OtherTok}[1]{\textcolor[rgb]{0.00,0.44,0.13}{{#1}}}
    \newcommand{\AlertTok}[1]{\textcolor[rgb]{1.00,0.00,0.00}{\textbf{{#1}}}}
    \newcommand{\FunctionTok}[1]{\textcolor[rgb]{0.02,0.16,0.49}{{#1}}}
    \newcommand{\RegionMarkerTok}[1]{{#1}}
    \newcommand{\ErrorTok}[1]{\textcolor[rgb]{1.00,0.00,0.00}{\textbf{{#1}}}}
    \newcommand{\NormalTok}[1]{{#1}}

    % Additional commands for more recent versions of Pandoc
    \newcommand{\ConstantTok}[1]{\textcolor[rgb]{0.53,0.00,0.00}{{#1}}}
    \newcommand{\SpecialCharTok}[1]{\textcolor[rgb]{0.25,0.44,0.63}{{#1}}}
    \newcommand{\VerbatimStringTok}[1]{\textcolor[rgb]{0.25,0.44,0.63}{{#1}}}
    \newcommand{\SpecialStringTok}[1]{\textcolor[rgb]{0.73,0.40,0.53}{{#1}}}
    \newcommand{\ImportTok}[1]{{#1}}
    \newcommand{\DocumentationTok}[1]{\textcolor[rgb]{0.73,0.13,0.13}{\textit{{#1}}}}
    \newcommand{\AnnotationTok}[1]{\textcolor[rgb]{0.38,0.63,0.69}{\textbf{\textit{{#1}}}}}
    \newcommand{\CommentVarTok}[1]{\textcolor[rgb]{0.38,0.63,0.69}{\textbf{\textit{{#1}}}}}
    \newcommand{\VariableTok}[1]{\textcolor[rgb]{0.10,0.09,0.49}{{#1}}}
    \newcommand{\ControlFlowTok}[1]{\textcolor[rgb]{0.00,0.44,0.13}{\textbf{{#1}}}}
    \newcommand{\OperatorTok}[1]{\textcolor[rgb]{0.40,0.40,0.40}{{#1}}}
    \newcommand{\BuiltInTok}[1]{{#1}}
    \newcommand{\ExtensionTok}[1]{{#1}}
    \newcommand{\PreprocessorTok}[1]{\textcolor[rgb]{0.74,0.48,0.00}{{#1}}}
    \newcommand{\AttributeTok}[1]{\textcolor[rgb]{0.49,0.56,0.16}{{#1}}}
    \newcommand{\InformationTok}[1]{\textcolor[rgb]{0.38,0.63,0.69}{\textbf{\textit{{#1}}}}}
    \newcommand{\WarningTok}[1]{\textcolor[rgb]{0.38,0.63,0.69}{\textbf{\textit{{#1}}}}}


    % Define a nice break command that doesn't care if a line doesn't already
    % exist.
    \def\br{\hspace*{\fill} \\* }
    % Math Jax compatibility definitions
    \def\gt{>}
    \def\lt{<}
    \let\Oldtex\TeX
    \let\Oldlatex\LaTeX
    \renewcommand{\TeX}{\textrm{\Oldtex}}
    \renewcommand{\LaTeX}{\textrm{\Oldlatex}}
    % Document parameters
    % Document title
    \title{HW\_1}
    
    
    
    
    
% Pygments definitions
\makeatletter
\def\PY@reset{\let\PY@it=\relax \let\PY@bf=\relax%
    \let\PY@ul=\relax \let\PY@tc=\relax%
    \let\PY@bc=\relax \let\PY@ff=\relax}
\def\PY@tok#1{\csname PY@tok@#1\endcsname}
\def\PY@toks#1+{\ifx\relax#1\empty\else%
    \PY@tok{#1}\expandafter\PY@toks\fi}
\def\PY@do#1{\PY@bc{\PY@tc{\PY@ul{%
    \PY@it{\PY@bf{\PY@ff{#1}}}}}}}
\def\PY#1#2{\PY@reset\PY@toks#1+\relax+\PY@do{#2}}

\expandafter\def\csname PY@tok@w\endcsname{\def\PY@tc##1{\textcolor[rgb]{0.73,0.73,0.73}{##1}}}
\expandafter\def\csname PY@tok@c\endcsname{\let\PY@it=\textit\def\PY@tc##1{\textcolor[rgb]{0.25,0.50,0.50}{##1}}}
\expandafter\def\csname PY@tok@cp\endcsname{\def\PY@tc##1{\textcolor[rgb]{0.74,0.48,0.00}{##1}}}
\expandafter\def\csname PY@tok@k\endcsname{\let\PY@bf=\textbf\def\PY@tc##1{\textcolor[rgb]{0.00,0.50,0.00}{##1}}}
\expandafter\def\csname PY@tok@kp\endcsname{\def\PY@tc##1{\textcolor[rgb]{0.00,0.50,0.00}{##1}}}
\expandafter\def\csname PY@tok@kt\endcsname{\def\PY@tc##1{\textcolor[rgb]{0.69,0.00,0.25}{##1}}}
\expandafter\def\csname PY@tok@o\endcsname{\def\PY@tc##1{\textcolor[rgb]{0.40,0.40,0.40}{##1}}}
\expandafter\def\csname PY@tok@ow\endcsname{\let\PY@bf=\textbf\def\PY@tc##1{\textcolor[rgb]{0.67,0.13,1.00}{##1}}}
\expandafter\def\csname PY@tok@nb\endcsname{\def\PY@tc##1{\textcolor[rgb]{0.00,0.50,0.00}{##1}}}
\expandafter\def\csname PY@tok@nf\endcsname{\def\PY@tc##1{\textcolor[rgb]{0.00,0.00,1.00}{##1}}}
\expandafter\def\csname PY@tok@nc\endcsname{\let\PY@bf=\textbf\def\PY@tc##1{\textcolor[rgb]{0.00,0.00,1.00}{##1}}}
\expandafter\def\csname PY@tok@nn\endcsname{\let\PY@bf=\textbf\def\PY@tc##1{\textcolor[rgb]{0.00,0.00,1.00}{##1}}}
\expandafter\def\csname PY@tok@ne\endcsname{\let\PY@bf=\textbf\def\PY@tc##1{\textcolor[rgb]{0.82,0.25,0.23}{##1}}}
\expandafter\def\csname PY@tok@nv\endcsname{\def\PY@tc##1{\textcolor[rgb]{0.10,0.09,0.49}{##1}}}
\expandafter\def\csname PY@tok@no\endcsname{\def\PY@tc##1{\textcolor[rgb]{0.53,0.00,0.00}{##1}}}
\expandafter\def\csname PY@tok@nl\endcsname{\def\PY@tc##1{\textcolor[rgb]{0.63,0.63,0.00}{##1}}}
\expandafter\def\csname PY@tok@ni\endcsname{\let\PY@bf=\textbf\def\PY@tc##1{\textcolor[rgb]{0.60,0.60,0.60}{##1}}}
\expandafter\def\csname PY@tok@na\endcsname{\def\PY@tc##1{\textcolor[rgb]{0.49,0.56,0.16}{##1}}}
\expandafter\def\csname PY@tok@nt\endcsname{\let\PY@bf=\textbf\def\PY@tc##1{\textcolor[rgb]{0.00,0.50,0.00}{##1}}}
\expandafter\def\csname PY@tok@nd\endcsname{\def\PY@tc##1{\textcolor[rgb]{0.67,0.13,1.00}{##1}}}
\expandafter\def\csname PY@tok@s\endcsname{\def\PY@tc##1{\textcolor[rgb]{0.73,0.13,0.13}{##1}}}
\expandafter\def\csname PY@tok@sd\endcsname{\let\PY@it=\textit\def\PY@tc##1{\textcolor[rgb]{0.73,0.13,0.13}{##1}}}
\expandafter\def\csname PY@tok@si\endcsname{\let\PY@bf=\textbf\def\PY@tc##1{\textcolor[rgb]{0.73,0.40,0.53}{##1}}}
\expandafter\def\csname PY@tok@se\endcsname{\let\PY@bf=\textbf\def\PY@tc##1{\textcolor[rgb]{0.73,0.40,0.13}{##1}}}
\expandafter\def\csname PY@tok@sr\endcsname{\def\PY@tc##1{\textcolor[rgb]{0.73,0.40,0.53}{##1}}}
\expandafter\def\csname PY@tok@ss\endcsname{\def\PY@tc##1{\textcolor[rgb]{0.10,0.09,0.49}{##1}}}
\expandafter\def\csname PY@tok@sx\endcsname{\def\PY@tc##1{\textcolor[rgb]{0.00,0.50,0.00}{##1}}}
\expandafter\def\csname PY@tok@m\endcsname{\def\PY@tc##1{\textcolor[rgb]{0.40,0.40,0.40}{##1}}}
\expandafter\def\csname PY@tok@gh\endcsname{\let\PY@bf=\textbf\def\PY@tc##1{\textcolor[rgb]{0.00,0.00,0.50}{##1}}}
\expandafter\def\csname PY@tok@gu\endcsname{\let\PY@bf=\textbf\def\PY@tc##1{\textcolor[rgb]{0.50,0.00,0.50}{##1}}}
\expandafter\def\csname PY@tok@gd\endcsname{\def\PY@tc##1{\textcolor[rgb]{0.63,0.00,0.00}{##1}}}
\expandafter\def\csname PY@tok@gi\endcsname{\def\PY@tc##1{\textcolor[rgb]{0.00,0.63,0.00}{##1}}}
\expandafter\def\csname PY@tok@gr\endcsname{\def\PY@tc##1{\textcolor[rgb]{1.00,0.00,0.00}{##1}}}
\expandafter\def\csname PY@tok@ge\endcsname{\let\PY@it=\textit}
\expandafter\def\csname PY@tok@gs\endcsname{\let\PY@bf=\textbf}
\expandafter\def\csname PY@tok@gp\endcsname{\let\PY@bf=\textbf\def\PY@tc##1{\textcolor[rgb]{0.00,0.00,0.50}{##1}}}
\expandafter\def\csname PY@tok@go\endcsname{\def\PY@tc##1{\textcolor[rgb]{0.53,0.53,0.53}{##1}}}
\expandafter\def\csname PY@tok@gt\endcsname{\def\PY@tc##1{\textcolor[rgb]{0.00,0.27,0.87}{##1}}}
\expandafter\def\csname PY@tok@err\endcsname{\def\PY@bc##1{\setlength{\fboxsep}{0pt}\fcolorbox[rgb]{1.00,0.00,0.00}{1,1,1}{\strut ##1}}}
\expandafter\def\csname PY@tok@kc\endcsname{\let\PY@bf=\textbf\def\PY@tc##1{\textcolor[rgb]{0.00,0.50,0.00}{##1}}}
\expandafter\def\csname PY@tok@kd\endcsname{\let\PY@bf=\textbf\def\PY@tc##1{\textcolor[rgb]{0.00,0.50,0.00}{##1}}}
\expandafter\def\csname PY@tok@kn\endcsname{\let\PY@bf=\textbf\def\PY@tc##1{\textcolor[rgb]{0.00,0.50,0.00}{##1}}}
\expandafter\def\csname PY@tok@kr\endcsname{\let\PY@bf=\textbf\def\PY@tc##1{\textcolor[rgb]{0.00,0.50,0.00}{##1}}}
\expandafter\def\csname PY@tok@bp\endcsname{\def\PY@tc##1{\textcolor[rgb]{0.00,0.50,0.00}{##1}}}
\expandafter\def\csname PY@tok@fm\endcsname{\def\PY@tc##1{\textcolor[rgb]{0.00,0.00,1.00}{##1}}}
\expandafter\def\csname PY@tok@vc\endcsname{\def\PY@tc##1{\textcolor[rgb]{0.10,0.09,0.49}{##1}}}
\expandafter\def\csname PY@tok@vg\endcsname{\def\PY@tc##1{\textcolor[rgb]{0.10,0.09,0.49}{##1}}}
\expandafter\def\csname PY@tok@vi\endcsname{\def\PY@tc##1{\textcolor[rgb]{0.10,0.09,0.49}{##1}}}
\expandafter\def\csname PY@tok@vm\endcsname{\def\PY@tc##1{\textcolor[rgb]{0.10,0.09,0.49}{##1}}}
\expandafter\def\csname PY@tok@sa\endcsname{\def\PY@tc##1{\textcolor[rgb]{0.73,0.13,0.13}{##1}}}
\expandafter\def\csname PY@tok@sb\endcsname{\def\PY@tc##1{\textcolor[rgb]{0.73,0.13,0.13}{##1}}}
\expandafter\def\csname PY@tok@sc\endcsname{\def\PY@tc##1{\textcolor[rgb]{0.73,0.13,0.13}{##1}}}
\expandafter\def\csname PY@tok@dl\endcsname{\def\PY@tc##1{\textcolor[rgb]{0.73,0.13,0.13}{##1}}}
\expandafter\def\csname PY@tok@s2\endcsname{\def\PY@tc##1{\textcolor[rgb]{0.73,0.13,0.13}{##1}}}
\expandafter\def\csname PY@tok@sh\endcsname{\def\PY@tc##1{\textcolor[rgb]{0.73,0.13,0.13}{##1}}}
\expandafter\def\csname PY@tok@s1\endcsname{\def\PY@tc##1{\textcolor[rgb]{0.73,0.13,0.13}{##1}}}
\expandafter\def\csname PY@tok@mb\endcsname{\def\PY@tc##1{\textcolor[rgb]{0.40,0.40,0.40}{##1}}}
\expandafter\def\csname PY@tok@mf\endcsname{\def\PY@tc##1{\textcolor[rgb]{0.40,0.40,0.40}{##1}}}
\expandafter\def\csname PY@tok@mh\endcsname{\def\PY@tc##1{\textcolor[rgb]{0.40,0.40,0.40}{##1}}}
\expandafter\def\csname PY@tok@mi\endcsname{\def\PY@tc##1{\textcolor[rgb]{0.40,0.40,0.40}{##1}}}
\expandafter\def\csname PY@tok@il\endcsname{\def\PY@tc##1{\textcolor[rgb]{0.40,0.40,0.40}{##1}}}
\expandafter\def\csname PY@tok@mo\endcsname{\def\PY@tc##1{\textcolor[rgb]{0.40,0.40,0.40}{##1}}}
\expandafter\def\csname PY@tok@ch\endcsname{\let\PY@it=\textit\def\PY@tc##1{\textcolor[rgb]{0.25,0.50,0.50}{##1}}}
\expandafter\def\csname PY@tok@cm\endcsname{\let\PY@it=\textit\def\PY@tc##1{\textcolor[rgb]{0.25,0.50,0.50}{##1}}}
\expandafter\def\csname PY@tok@cpf\endcsname{\let\PY@it=\textit\def\PY@tc##1{\textcolor[rgb]{0.25,0.50,0.50}{##1}}}
\expandafter\def\csname PY@tok@c1\endcsname{\let\PY@it=\textit\def\PY@tc##1{\textcolor[rgb]{0.25,0.50,0.50}{##1}}}
\expandafter\def\csname PY@tok@cs\endcsname{\let\PY@it=\textit\def\PY@tc##1{\textcolor[rgb]{0.25,0.50,0.50}{##1}}}

\def\PYZbs{\char`\\}
\def\PYZus{\char`\_}
\def\PYZob{\char`\{}
\def\PYZcb{\char`\}}
\def\PYZca{\char`\^}
\def\PYZam{\char`\&}
\def\PYZlt{\char`\<}
\def\PYZgt{\char`\>}
\def\PYZsh{\char`\#}
\def\PYZpc{\char`\%}
\def\PYZdl{\char`\$}
\def\PYZhy{\char`\-}
\def\PYZsq{\char`\'}
\def\PYZdq{\char`\"}
\def\PYZti{\char`\~}
% for compatibility with earlier versions
\def\PYZat{@}
\def\PYZlb{[}
\def\PYZrb{]}
\makeatother


    % For linebreaks inside Verbatim environment from package fancyvrb.
    \makeatletter
        \newbox\Wrappedcontinuationbox
        \newbox\Wrappedvisiblespacebox
        \newcommand*\Wrappedvisiblespace {\textcolor{red}{\textvisiblespace}}
        \newcommand*\Wrappedcontinuationsymbol {\textcolor{red}{\llap{\tiny$\m@th\hookrightarrow$}}}
        \newcommand*\Wrappedcontinuationindent {3ex }
        \newcommand*\Wrappedafterbreak {\kern\Wrappedcontinuationindent\copy\Wrappedcontinuationbox}
        % Take advantage of the already applied Pygments mark-up to insert
        % potential linebreaks for TeX processing.
        %        {, <, #, %, $, ' and ": go to next line.
        %        _, }, ^, &, >, - and ~: stay at end of broken line.
        % Use of \textquotesingle for straight quote.
        \newcommand*\Wrappedbreaksatspecials {%
            \def\PYGZus{\discretionary{\char`\_}{\Wrappedafterbreak}{\char`\_}}%
            \def\PYGZob{\discretionary{}{\Wrappedafterbreak\char`\{}{\char`\{}}%
            \def\PYGZcb{\discretionary{\char`\}}{\Wrappedafterbreak}{\char`\}}}%
            \def\PYGZca{\discretionary{\char`\^}{\Wrappedafterbreak}{\char`\^}}%
            \def\PYGZam{\discretionary{\char`\&}{\Wrappedafterbreak}{\char`\&}}%
            \def\PYGZlt{\discretionary{}{\Wrappedafterbreak\char`\<}{\char`\<}}%
            \def\PYGZgt{\discretionary{\char`\>}{\Wrappedafterbreak}{\char`\>}}%
            \def\PYGZsh{\discretionary{}{\Wrappedafterbreak\char`\#}{\char`\#}}%
            \def\PYGZpc{\discretionary{}{\Wrappedafterbreak\char`\%}{\char`\%}}%
            \def\PYGZdl{\discretionary{}{\Wrappedafterbreak\char`\$}{\char`\$}}%
            \def\PYGZhy{\discretionary{\char`\-}{\Wrappedafterbreak}{\char`\-}}%
            \def\PYGZsq{\discretionary{}{\Wrappedafterbreak\textquotesingle}{\textquotesingle}}%
            \def\PYGZdq{\discretionary{}{\Wrappedafterbreak\char`\"}{\char`\"}}%
            \def\PYGZti{\discretionary{\char`\~}{\Wrappedafterbreak}{\char`\~}}%
        }
        % Some characters . , ; ? ! / are not pygmentized.
        % This macro makes them "active" and they will insert potential linebreaks
        \newcommand*\Wrappedbreaksatpunct {%
            \lccode`\~`\.\lowercase{\def~}{\discretionary{\hbox{\char`\.}}{\Wrappedafterbreak}{\hbox{\char`\.}}}%
            \lccode`\~`\,\lowercase{\def~}{\discretionary{\hbox{\char`\,}}{\Wrappedafterbreak}{\hbox{\char`\,}}}%
            \lccode`\~`\;\lowercase{\def~}{\discretionary{\hbox{\char`\;}}{\Wrappedafterbreak}{\hbox{\char`\;}}}%
            \lccode`\~`\:\lowercase{\def~}{\discretionary{\hbox{\char`\:}}{\Wrappedafterbreak}{\hbox{\char`\:}}}%
            \lccode`\~`\?\lowercase{\def~}{\discretionary{\hbox{\char`\?}}{\Wrappedafterbreak}{\hbox{\char`\?}}}%
            \lccode`\~`\!\lowercase{\def~}{\discretionary{\hbox{\char`\!}}{\Wrappedafterbreak}{\hbox{\char`\!}}}%
            \lccode`\~`\/\lowercase{\def~}{\discretionary{\hbox{\char`\/}}{\Wrappedafterbreak}{\hbox{\char`\/}}}%
            \catcode`\.\active
            \catcode`\,\active
            \catcode`\;\active
            \catcode`\:\active
            \catcode`\?\active
            \catcode`\!\active
            \catcode`\/\active
            \lccode`\~`\~
        }
    \makeatother

    \let\OriginalVerbatim=\Verbatim
    \makeatletter
    \renewcommand{\Verbatim}[1][1]{%
        %\parskip\z@skip
        \sbox\Wrappedcontinuationbox {\Wrappedcontinuationsymbol}%
        \sbox\Wrappedvisiblespacebox {\FV@SetupFont\Wrappedvisiblespace}%
        \def\FancyVerbFormatLine ##1{\hsize\linewidth
            \vtop{\raggedright\hyphenpenalty\z@\exhyphenpenalty\z@
                \doublehyphendemerits\z@\finalhyphendemerits\z@
                \strut ##1\strut}%
        }%
        % If the linebreak is at a space, the latter will be displayed as visible
        % space at end of first line, and a continuation symbol starts next line.
        % Stretch/shrink are however usually zero for typewriter font.
        \def\FV@Space {%
            \nobreak\hskip\z@ plus\fontdimen3\font minus\fontdimen4\font
            \discretionary{\copy\Wrappedvisiblespacebox}{\Wrappedafterbreak}
            {\kern\fontdimen2\font}%
        }%

        % Allow breaks at special characters using \PYG... macros.
        \Wrappedbreaksatspecials
        % Breaks at punctuation characters . , ; ? ! and / need catcode=\active
        \OriginalVerbatim[#1,codes*=\Wrappedbreaksatpunct]%
    }
    \makeatother

    % Exact colors from NB
    \definecolor{incolor}{HTML}{303F9F}
    \definecolor{outcolor}{HTML}{D84315}
    \definecolor{cellborder}{HTML}{CFCFCF}
    \definecolor{cellbackground}{HTML}{F7F7F7}

    % prompt
    \makeatletter
    \newcommand{\boxspacing}{\kern\kvtcb@left@rule\kern\kvtcb@boxsep}
    \makeatother
    \newcommand{\prompt}[4]{
        {\ttfamily\llap{{\color{#2}[#3]:\hspace{3pt}#4}}\vspace{-\baselineskip}}
    }
    

    
    % Prevent overflowing lines due to hard-to-break entities
    \sloppy
    % Setup hyperref package
    \hypersetup{
      breaklinks=true,  % so long urls are correctly broken across lines
      colorlinks=true,
      urlcolor=urlcolor,
      linkcolor=linkcolor,
      citecolor=citecolor,
      }
    % Slightly bigger margins than the latex defaults
    
    \geometry{verbose,tmargin=1in,bmargin=1in,lmargin=1in,rmargin=1in}
    
    

\begin{document}
    
    \maketitle
    
    

    
    \hypertarget{ee634-hw1}{%
\section{EE634 HW1}\label{ee634-hw1}}

\hypertarget{kutay-ugurlu}{%
\subsubsection{Kutay Ugurlu}\label{kutay-ugurlu}}

    \begin{tcolorbox}[breakable, size=fbox, boxrule=1pt, pad at break*=1mm,colback=cellbackground, colframe=cellborder]
\prompt{In}{incolor}{11}{\boxspacing}
\begin{Verbatim}[commandchars=\\\{\}]
\PY{k+kn}{import} \PY{n+nn}{numpy} \PY{k}{as} \PY{n+nn}{np} 
\PY{k+kn}{from} \PY{n+nn}{scipy}\PY{n+nn}{.}\PY{n+nn}{fft} \PY{k+kn}{import} \PY{n}{fft}\PY{p}{,} \PY{n}{ifft}\PY{p}{,} \PY{n}{fft2}\PY{p}{,} \PY{n}{ifft2}\PY{p}{,} \PY{n}{fftshift}
\PY{k+kn}{from} \PY{n+nn}{scipy}\PY{n+nn}{.}\PY{n+nn}{signal} \PY{k+kn}{import} \PY{n}{convolve2d}
\PY{k+kn}{from} \PY{n+nn}{scipy}\PY{n+nn}{.}\PY{n+nn}{linalg} \PY{k+kn}{import} \PY{n}{toeplitz}
\PY{k+kn}{from} \PY{n+nn}{matplotlib} \PY{k+kn}{import} \PY{n}{pyplot} \PY{k}{as} \PY{n}{plt}
\PY{k+kn}{from} \PY{n+nn}{matplotlib} \PY{k+kn}{import} \PY{n}{cm}
\PY{k+kn}{from} \PY{n+nn}{numpy} \PY{k+kn}{import} \PY{n}{pi} \PY{k}{as} \PY{n}{pi} 
\PY{k+kn}{from} \PY{n+nn}{mpl\PYZus{}toolkits}\PY{n+nn}{.}\PY{n+nn}{mplot3d} \PY{k+kn}{import} \PY{n}{Axes3D}
\PY{o}{\PYZpc{}}\PY{k}{matplotlib} inline
\end{Verbatim}
\end{tcolorbox}

    \hypertarget{q1a}{%
\subsection{Q1a}\label{q1a}}

    \[x(n_1,n_2) = \begin{bmatrix} 1 & 1\\  1 & 1 \end{bmatrix}\]
\[x\ast h(n_1,n_2) = \begin{bmatrix} 0 & \frac{1}{4} & \frac{1}{4} & 0 \\  \frac{1}{4} & 1 & 1 & \frac{1}{4} \\ \frac{1}{4} & 1 & 1 & \frac{1}{4} \\ 0 & \frac{1}{4} & \frac{1}{4} & 0 \end{bmatrix}\]

Using the linear convolutions dimension expression one can conclude that
the filter is \(3\times 3\). So let
\[h(-n_1,-n_2) = \begin{bmatrix} a & b & c \\  d & e & f \\ g & h & i \end{bmatrix}\]

Using the corner elements, one can deduce that the corner elements of
the filter is 0. With this configuration: * \(i = 0\) *
\(h + i = \frac{1}{4} \implies h = \frac{1}{4}\) *
\(g + h = \frac{1}{4} \implies g = 0\) *
\(f + i = \frac{1}{4} \implies f = \frac{1}{4}\) *
\(e + h + f + i = 1 \implies e = \frac{1}{2}\) *
\(d + g + e + h = 1 \implies d = \frac{1}{4}\) \(\vdots\)

By using the symmetry in input and output, one can also conclude that: *
\(a = c = 0\) * b = \(\frac{1}{4}\)

Hence
\[h(n_1,n_2) = \begin{bmatrix} 0 & \frac{1}{4} & 0 \\  \frac{1}{4} & \frac{1}{2} & \frac{1}{4} \\ 0 & \frac{1}{4} & 0 \end{bmatrix}\]

    \begin{tcolorbox}[breakable, size=fbox, boxrule=1pt, pad at break*=1mm,colback=cellbackground, colframe=cellborder]
\prompt{In}{incolor}{12}{\boxspacing}
\begin{Verbatim}[commandchars=\\\{\}]
\PY{n}{x} \PY{o}{=} \PY{n}{np}\PY{o}{.}\PY{n}{zeros}\PY{p}{(}\PY{p}{(}\PY{l+m+mi}{8}\PY{p}{,}\PY{l+m+mi}{8}\PY{p}{)}\PY{p}{)}
\PY{n}{x}\PY{p}{[}\PY{l+m+mi}{3}\PY{p}{:}\PY{l+m+mi}{5}\PY{p}{,}\PY{l+m+mi}{3}\PY{p}{:}\PY{l+m+mi}{5}\PY{p}{]} \PY{o}{=} \PY{l+m+mi}{1}
\PY{n}{y} \PY{o}{=} \PY{n}{np}\PY{o}{.}\PY{n}{zeros\PYZus{}like}\PY{p}{(}\PY{n}{x}\PY{p}{)}
\PY{n}{y}\PY{p}{[}\PY{l+m+mi}{2}\PY{p}{:}\PY{l+m+mi}{6}\PY{p}{,}\PY{l+m+mi}{2}\PY{p}{:}\PY{l+m+mi}{6}\PY{p}{]} \PY{o}{=} \PY{n}{np}\PY{o}{.}\PY{n}{array}\PY{p}{(}\PY{p}{[}\PY{p}{[}\PY{l+m+mi}{0}\PY{p}{,}\PY{o}{.}\PY{l+m+mi}{25}\PY{p}{,}\PY{o}{.}\PY{l+m+mi}{25}\PY{p}{,}\PY{l+m+mi}{0}\PY{p}{]}\PY{p}{,}\PY{p}{[}\PY{o}{.}\PY{l+m+mi}{25}\PY{p}{,}\PY{l+m+mi}{1}\PY{p}{,}\PY{l+m+mi}{1}\PY{p}{,}\PY{o}{.}\PY{l+m+mi}{25}\PY{p}{]}\PY{p}{,}\PY{p}{[}\PY{o}{.}\PY{l+m+mi}{25}\PY{p}{,}\PY{l+m+mi}{1}\PY{p}{,}\PY{l+m+mi}{1}\PY{p}{,}\PY{o}{.}\PY{l+m+mi}{25}\PY{p}{]}\PY{p}{,}\PY{p}{[}\PY{l+m+mi}{0}\PY{p}{,}\PY{o}{.}\PY{l+m+mi}{25}\PY{p}{,}\PY{o}{.}\PY{l+m+mi}{25}\PY{p}{,}\PY{l+m+mi}{0}\PY{p}{]}\PY{p}{]}\PY{p}{)}
\PY{n}{h} \PY{o}{=} \PY{n}{np}\PY{o}{.}\PY{n}{zeros\PYZus{}like}\PY{p}{(}\PY{n}{x}\PY{p}{)}
\PY{n}{h}\PY{p}{[}\PY{l+m+mi}{3}\PY{p}{:}\PY{l+m+mi}{6}\PY{p}{,}\PY{l+m+mi}{3}\PY{p}{:}\PY{l+m+mi}{6}\PY{p}{]} \PY{o}{=} \PY{n}{np}\PY{o}{.}\PY{n}{array}\PY{p}{(}\PY{p}{[}\PY{p}{[}\PY{l+m+mi}{0}\PY{p}{,}\PY{o}{.}\PY{l+m+mi}{25}\PY{p}{,}\PY{l+m+mi}{0}\PY{p}{]}\PY{p}{,}\PY{p}{[}\PY{o}{.}\PY{l+m+mi}{25}\PY{p}{,}\PY{o}{.}\PY{l+m+mi}{5}\PY{p}{,}\PY{o}{.}\PY{l+m+mi}{25}\PY{p}{]}\PY{p}{,}\PY{p}{[}\PY{l+m+mi}{0}\PY{p}{,}\PY{o}{.}\PY{l+m+mi}{25}\PY{p}{,}\PY{l+m+mi}{0}\PY{p}{]}\PY{p}{]}\PY{p}{)}
\PY{n}{y\PYZus{}prime} \PY{o}{=} \PY{n}{np}\PY{o}{.}\PY{n}{abs}\PY{p}{(}\PY{n}{fftshift}\PY{p}{(}\PY{n}{ifft2}\PY{p}{(}\PY{n}{fft2}\PY{p}{(}\PY{n}{x}\PY{p}{)}\PY{o}{*}\PY{n}{fft2}\PY{p}{(}\PY{n}{h}\PY{p}{)}\PY{p}{)}\PY{p}{)}\PY{p}{)}
\PY{n}{fsize} \PY{o}{=} \PY{l+m+mi}{40}
\PY{n}{plt}\PY{o}{.}\PY{n}{figure}\PY{p}{(}\PY{n}{figsize}\PY{o}{=}\PY{p}{(}\PY{l+m+mi}{20}\PY{p}{,}\PY{l+m+mi}{80}\PY{p}{)}\PY{p}{)}
\PY{n}{plt}\PY{o}{.}\PY{n}{subplot}\PY{p}{(}\PY{l+m+mi}{1}\PY{p}{,}\PY{l+m+mi}{4}\PY{p}{,}\PY{l+m+mi}{1}\PY{p}{)}
\PY{n}{plt}\PY{o}{.}\PY{n}{imshow}\PY{p}{(}\PY{n}{np}\PY{o}{.}\PY{n}{abs}\PY{p}{(}\PY{n}{x}\PY{p}{)}\PY{p}{)}
\PY{n}{plt}\PY{o}{.}\PY{n}{title}\PY{p}{(}\PY{l+s+s1}{\PYZsq{}}\PY{l+s+s1}{X}\PY{l+s+s1}{\PYZsq{}}\PY{p}{,}\PY{n}{fontsize}\PY{o}{=}\PY{n}{fsize}\PY{p}{)}
\PY{n}{plt}\PY{o}{.}\PY{n}{set\PYZus{}cmap}\PY{p}{(}\PY{n}{cmap}\PY{o}{=}\PY{l+s+s2}{\PYZdq{}}\PY{l+s+s2}{gray}\PY{l+s+s2}{\PYZdq{}}\PY{p}{)}
\PY{n}{plt}\PY{o}{.}\PY{n}{subplot}\PY{p}{(}\PY{l+m+mi}{1}\PY{p}{,}\PY{l+m+mi}{4}\PY{p}{,}\PY{l+m+mi}{2}\PY{p}{)}
\PY{n}{plt}\PY{o}{.}\PY{n}{imshow}\PY{p}{(}\PY{n}{np}\PY{o}{.}\PY{n}{abs}\PY{p}{(}\PY{n}{y}\PY{p}{)}\PY{p}{)}
\PY{n}{plt}\PY{o}{.}\PY{n}{title}\PY{p}{(}\PY{l+s+s1}{\PYZsq{}}\PY{l+s+s1}{Y}\PY{l+s+s1}{\PYZsq{}}\PY{p}{,}\PY{n}{fontsize}\PY{o}{=}\PY{n}{fsize}\PY{p}{)}
\PY{n}{plt}\PY{o}{.}\PY{n}{set\PYZus{}cmap}\PY{p}{(}\PY{n}{cmap}\PY{o}{=}\PY{l+s+s2}{\PYZdq{}}\PY{l+s+s2}{gray}\PY{l+s+s2}{\PYZdq{}}\PY{p}{)}
\PY{n}{plt}\PY{o}{.}\PY{n}{subplot}\PY{p}{(}\PY{l+m+mi}{1}\PY{p}{,}\PY{l+m+mi}{4}\PY{p}{,}\PY{l+m+mi}{3}\PY{p}{)}
\PY{n}{plt}\PY{o}{.}\PY{n}{imshow}\PY{p}{(}\PY{n}{np}\PY{o}{.}\PY{n}{abs}\PY{p}{(}\PY{n}{h}\PY{p}{)}\PY{p}{)}
\PY{n}{plt}\PY{o}{.}\PY{n}{title}\PY{p}{(}\PY{l+s+s1}{\PYZsq{}}\PY{l+s+s1}{H}\PY{l+s+s1}{\PYZsq{}}\PY{p}{,}\PY{n}{fontsize}\PY{o}{=}\PY{n}{fsize}\PY{p}{)}
\PY{n}{plt}\PY{o}{.}\PY{n}{set\PYZus{}cmap}\PY{p}{(}\PY{n}{cmap}\PY{o}{=}\PY{l+s+s2}{\PYZdq{}}\PY{l+s+s2}{gray}\PY{l+s+s2}{\PYZdq{}}\PY{p}{)}
\PY{n}{plt}\PY{o}{.}\PY{n}{subplot}\PY{p}{(}\PY{l+m+mi}{1}\PY{p}{,}\PY{l+m+mi}{4}\PY{p}{,}\PY{l+m+mi}{4}\PY{p}{)}
\PY{n}{plt}\PY{o}{.}\PY{n}{imshow}\PY{p}{(}\PY{n}{y\PYZus{}prime}\PY{p}{)}
\PY{n}{plt}\PY{o}{.}\PY{n}{title}\PY{p}{(}\PY{l+s+sa}{r}\PY{l+s+s1}{\PYZsq{}}\PY{l+s+s1}{H \PYZdl{}}\PY{l+s+s1}{\PYZbs{}}\PY{l+s+s1}{ast\PYZdl{} X}\PY{l+s+s1}{\PYZsq{}}\PY{p}{,}\PY{n}{fontsize}\PY{o}{=}\PY{n}{fsize}\PY{p}{)}
\PY{n}{plt}\PY{o}{.}\PY{n}{set\PYZus{}cmap}\PY{p}{(}\PY{n}{cmap}\PY{o}{=}\PY{l+s+s2}{\PYZdq{}}\PY{l+s+s2}{gray}\PY{l+s+s2}{\PYZdq{}}\PY{p}{)}
\PY{k}{assert} \PY{n}{np}\PY{o}{.}\PY{n}{isclose}\PY{p}{(}\PY{n}{np}\PY{o}{.}\PY{n}{sum}\PY{p}{(}\PY{n}{y\PYZus{}prime}\PY{o}{\PYZhy{}}\PY{n}{y}\PY{p}{)}\PY{p}{,}\PY{l+m+mi}{0}\PY{p}{)} 
\PY{n+nb}{print}\PY{p}{(}\PY{l+s+s2}{\PYZdq{}}\PY{l+s+s2}{Resultant convolution matches the given.}\PY{l+s+s2}{\PYZdq{}}\PY{p}{)}
\end{Verbatim}
\end{tcolorbox}

    \begin{Verbatim}[commandchars=\\\{\}]
Resultant convolution matches the given.
    \end{Verbatim}

    \begin{center}
    \adjustimage{max size={0.9\linewidth}{0.9\paperheight}}{HW_1_files/HW_1_4_1.png}
    \end{center}
    { \hspace*{\fill} \\}
    
    \hypertarget{q1b}{%
\section{Q1b}\label{q1b}}

    \begin{tcolorbox}[breakable, size=fbox, boxrule=1pt, pad at break*=1mm,colback=cellbackground, colframe=cellborder]
\prompt{In}{incolor}{13}{\boxspacing}
\begin{Verbatim}[commandchars=\\\{\}]
\PY{n}{H} \PY{o}{=} \PY{n}{h}\PY{p}{[}\PY{l+m+mi}{3}\PY{p}{:}\PY{l+m+mi}{6}\PY{p}{,}\PY{l+m+mi}{3}\PY{p}{:}\PY{l+m+mi}{6}\PY{p}{]}
\PY{n}{H}
\end{Verbatim}
\end{tcolorbox}

            \begin{tcolorbox}[breakable, size=fbox, boxrule=.5pt, pad at break*=1mm, opacityfill=0]
\prompt{Out}{outcolor}{13}{\boxspacing}
\begin{Verbatim}[commandchars=\\\{\}]
array([[0.  , 0.25, 0.  ],
       [0.25, 0.5 , 0.25],
       [0.  , 0.25, 0.  ]])
\end{Verbatim}
\end{tcolorbox}
        
    \begin{tcolorbox}[breakable, size=fbox, boxrule=1pt, pad at break*=1mm,colback=cellbackground, colframe=cellborder]
\prompt{In}{incolor}{14}{\boxspacing}
\begin{Verbatim}[commandchars=\\\{\}]
\PY{n}{u} \PY{o}{=} \PY{n}{np}\PY{o}{.}\PY{n}{linalg}\PY{o}{.}\PY{n}{svd}\PY{p}{(}\PY{n}{H}\PY{p}{)}
\PY{n}{singular\PYZus{}values} \PY{o}{=} \PY{n}{u}\PY{p}{[}\PY{l+m+mi}{1}\PY{p}{]}
\PY{n}{singular\PYZus{}values}
\end{Verbatim}
\end{tcolorbox}

            \begin{tcolorbox}[breakable, size=fbox, boxrule=.5pt, pad at break*=1mm, opacityfill=0]
\prompt{Out}{outcolor}{14}{\boxspacing}
\begin{Verbatim}[commandchars=\\\{\}]
array([6.83012702e-01, 1.83012702e-01, 8.77708367e-18])
\end{Verbatim}
\end{tcolorbox}
        
    \hypertarget{as-can-be-seen-above-we-have-3-distinct-singular-values-for-the-convolution-kernel.-for-a-kernel-to-be-separable-it-should-be-expressed-as-one-outer-product.-however-when-we-use-the-svd-to-decompose-the-matrix-into-outer-products-we-see-that-it-has-two-nonzero-singular-values.-one-separable-filter-example-can-be-seen-below.-the-kernel-is-defined-as-an-outer-product-and-it-has-only-one-nonzero-singular-value.}{%
\subsection{As can be seen above, we have 3 distinct singular values for
the convolution kernel. For a kernel to be separable, it should be
expressed as one outer product. However, when we use the SVD to
decompose the matrix into outer products, we see that it has two nonzero
singular values. One separable filter example can be seen below. The
kernel is defined as an outer product and it has only one nonzero
singular
value.}\label{as-can-be-seen-above-we-have-3-distinct-singular-values-for-the-convolution-kernel.-for-a-kernel-to-be-separable-it-should-be-expressed-as-one-outer-product.-however-when-we-use-the-svd-to-decompose-the-matrix-into-outer-products-we-see-that-it-has-two-nonzero-singular-values.-one-separable-filter-example-can-be-seen-below.-the-kernel-is-defined-as-an-outer-product-and-it-has-only-one-nonzero-singular-value.}}

    \begin{tcolorbox}[breakable, size=fbox, boxrule=1pt, pad at break*=1mm,colback=cellbackground, colframe=cellborder]
\prompt{In}{incolor}{15}{\boxspacing}
\begin{Verbatim}[commandchars=\\\{\}]
\PY{n}{a} \PY{o}{=} \PY{n}{np}\PY{o}{.}\PY{n}{random}\PY{o}{.}\PY{n}{randint}\PY{p}{(}\PY{l+m+mi}{0}\PY{p}{,}\PY{l+m+mi}{100}\PY{p}{,}\PY{p}{(}\PY{l+m+mi}{3}\PY{p}{,}\PY{l+m+mi}{1}\PY{p}{)}\PY{p}{)}
\PY{n}{b} \PY{o}{=} \PY{n}{np}\PY{o}{.}\PY{n}{outer}\PY{p}{(}\PY{n}{a}\PY{p}{,}\PY{n}{a}\PY{p}{)}
\PY{n}{u} \PY{o}{=} \PY{n}{np}\PY{o}{.}\PY{n}{linalg}\PY{o}{.}\PY{n}{svd}\PY{p}{(}\PY{n}{b}\PY{p}{)}
\PY{n}{singular\PYZus{}values} \PY{o}{=} \PY{n}{u}\PY{p}{[}\PY{l+m+mi}{1}\PY{p}{]}
\PY{n+nb}{print}\PY{p}{(}\PY{n}{singular\PYZus{}values}\PY{p}{)}
\PY{n}{plt}\PY{o}{.}\PY{n}{imshow}\PY{p}{(}\PY{n}{b}\PY{p}{)}
\end{Verbatim}
\end{tcolorbox}

    \begin{Verbatim}[commandchars=\\\{\}]
[11590.     0.     0.]
    \end{Verbatim}

            \begin{tcolorbox}[breakable, size=fbox, boxrule=.5pt, pad at break*=1mm, opacityfill=0]
\prompt{Out}{outcolor}{15}{\boxspacing}
\begin{Verbatim}[commandchars=\\\{\}]
<matplotlib.image.AxesImage at 0x25fe7dfc820>
\end{Verbatim}
\end{tcolorbox}
        
    \begin{center}
    \adjustimage{max size={0.9\linewidth}{0.9\paperheight}}{HW_1_files/HW_1_9_2.png}
    \end{center}
    { \hspace*{\fill} \\}
    
    \hypertarget{q1c}{%
\section{Q1c}\label{q1c}}

\(\begin{align} H(w_1,w_2) &= \sum\limits_{n_1=0}^2\sum\limits_{n_2=0}^2 h(n_1,n_2) e^{-jw_1n_1} e^{-jw_2n_2} \\  &= \frac{1}{4}(e^{-jw_1}+e^{-jw_2}+e^{-jw_1}e^{-j2w_2}+e^{-j2w_1}e^{-jw_2}) + \frac{1}{4}(e^{-jw_1}e^{-jw_2}) \nonumber\\  &= \frac{1}{4}e^{-jw_1}(1+e^{-j2w_2}) + \frac{1}{4}e^{-jw_2}(1+e^{-j2w_1}) + \frac{1}{2}(e^{-jw_1}e^{-jw_2}) \nonumber\\  &= \frac{1}{4}e^{-jw_1}e^{-jw_2}(cos(\frac{w_1}{2})+cos(\frac{w_2}{2})) + \frac{1}{2} e^{-jw_1}e^{-jw_2} \nonumber\\ \end{align}\)
\[\implies |H(w1,w2)| = \frac{1}{2} + cos(\frac{w_1}{2})+cos(\frac{w_2}{2})\]
This filter acts as low pass filter, since its magnitude have higher
values around the origin.

    \hypertarget{q1c}{%
\section{Q1c}\label{q1c}}

\(\begin{align} H(k_1,k_2) &= \sum\limits_{n_1=0}^2\sum\limits_{n_2=0}^2 h(n_1,n_2) e^{-j{\frac{2\pi}{N_1}}k_1n_1} e^{-j{\frac{2\pi}{N_2}}k_2n_2} \\ \end{align}\)
Since image has n1-n2 symmetry, its DFT has k1-k2 symmetry in the
frequency domain, \emph{i.e.} \(H(k_1,k_2) = H(k_2,k_1)\)

\(\begin{align} H(0,0) &= \sum\limits_{n_1=0}^2\sum\limits_{n_2=0}^2 h(n_1,n_2) = 1.5 \tag{1}\\ H(1,0) &= \sum\limits_{n_1=0}^2\sum\limits_{n_2=0}^2 h(n_1,n_2) e^{-j{\frac{2\pi n_1}{3}}} \tag{2}\\ &= (h(0,1)+h(0,2)+h(0,3)) + (h(1,0)+h(1,1)+h(1,2))e^{-j\frac{2\pi}{3}} + (h(2,1)+h(2,2)+h(3,2))e^{-j\frac{4\pi}{3}} \nonumber\\ &= \frac{1}{4} + e^{-j\frac{2\pi}{3}} + \frac{1}{4}e^{-j\frac{4\pi}{3}} \nonumber\\ H(0,1) &= H(1,0) \tag{3}\\ H(2,0) &= \sum\limits_{n_1=0}^2\sum\limits_{n_2=0}^2 h(n_1,n_2) e^{-j{\frac{4\pi n_1}{3}}} \tag{4}\\ &= (h(0,1)+h(0,2)+h(0,3)) + (h(1,0)+h(1,1)+h(1,2))e^{-j\frac{4\pi}{3}} + (h(2,1)+h(2,2)+h(3,2))e^{-j\frac{8\pi}{3}} \nonumber\\ &= \frac{1}{4} + e^{-j\frac{4\pi}{3}} + \frac{1}{4}e^{-j\frac{2\pi}{3}} \nonumber\\ H(0,2) &= H(2,0) \tag{5}\\ H(1,1) &= \sum\limits_{n_1=0}^2\sum\limits_{n_2=0}^2 h(n_1,n_2) e^{-j{\frac{2\pi (n_1+n_2)}{3}}} \tag{6}\\ &= \sum\limits_{n_2=0}^2 e^{-j{\frac{2\pi n_2}{3}}} \left( h(0,n_2) + h(1,n_2) e^{-j{\frac{2\pi}{3}}} + h(2,n_2) e^{-j{\frac{4\pi}{3}}} \right) \nonumber \\ &= 0 \ \ (See \ calculation \ below)\nonumber \\ H(2,2) &= \sum\limits_{n_1=0}^2\sum\limits_{n_2=0}^2 h(n_1,n_2) e^{-j{\frac{2\pi (2n_1+2n_2)}{3}}} \tag{7}\\ &= \sum\limits_{n_2=0}^2 e^{-j{\frac{4\pi n_2}{3}}} \left( h(0,n_2) + h(1,n_2) e^{-j{\frac{2\pi}{3}}} + h(2,n_2) e^{-j{\frac{4\pi}{3}}} \right) \nonumber \\ &= 0 \ \ (See \ calculation \ below)\nonumber \\ \end{align}\)

    \begin{tcolorbox}[breakable, size=fbox, boxrule=1pt, pad at break*=1mm,colback=cellbackground, colframe=cellborder]
\prompt{In}{incolor}{16}{\boxspacing}
\begin{Verbatim}[commandchars=\\\{\}]
\PY{n+nb}{sum} \PY{o}{=} \PY{l+m+mi}{0}
\PY{k}{for} \PY{n}{n2} \PY{o+ow}{in} \PY{n+nb}{range}\PY{p}{(}\PY{l+m+mi}{3}\PY{p}{)}\PY{p}{:}
    \PY{n+nb}{sum} \PY{o}{+}\PY{o}{=} \PY{n}{np}\PY{o}{.}\PY{n}{exp}\PY{p}{(}\PY{o}{\PYZhy{}}\PY{l+m+mi}{1}\PY{n}{j}\PY{o}{*}\PY{l+m+mi}{2}\PY{o}{*}\PY{n}{pi}\PY{o}{*}\PY{n}{n2}\PY{o}{/}\PY{l+m+mi}{3}\PY{p}{)} \PY{o}{*} \PY{p}{(}\PY{n}{h}\PY{p}{[}\PY{l+m+mi}{0}\PY{p}{,}\PY{n}{n2}\PY{p}{]} \PY{o}{+} \PY{n}{h}\PY{p}{[}\PY{l+m+mi}{1}\PY{p}{,}\PY{n}{n2}\PY{p}{]} \PY{o}{*} \PY{n}{np}\PY{o}{.}\PY{n}{exp}\PY{p}{(}\PY{o}{\PYZhy{}}\PY{l+m+mi}{1}\PY{n}{j}\PY{o}{*}\PY{l+m+mi}{2}\PY{o}{*}\PY{n}{pi}\PY{o}{/}\PY{l+m+mi}{3}\PY{p}{)} \PY{o}{+} \PY{n}{h}\PY{p}{[}\PY{l+m+mi}{2}\PY{p}{,}\PY{n}{n2}\PY{p}{]} \PY{o}{*} \PY{n}{np}\PY{o}{.}\PY{n}{exp}\PY{p}{(}\PY{o}{\PYZhy{}}\PY{l+m+mi}{1}\PY{n}{j}\PY{o}{*}\PY{l+m+mi}{4}\PY{o}{*}\PY{n}{pi}\PY{o}{/}\PY{l+m+mi}{3}\PY{p}{)}\PY{p}{)}
\PY{n+nb}{print}\PY{p}{(}\PY{l+s+s2}{\PYZdq{}}\PY{l+s+s2}{H(1,1) =}\PY{l+s+s2}{\PYZdq{}}\PY{p}{,}\PY{n+nb}{sum}\PY{p}{)}
\PY{n+nb}{sum} \PY{o}{=} \PY{l+m+mi}{0}
\PY{k}{for} \PY{n}{n2} \PY{o+ow}{in} \PY{n+nb}{range}\PY{p}{(}\PY{l+m+mi}{3}\PY{p}{)}\PY{p}{:}
    \PY{n+nb}{sum} \PY{o}{+}\PY{o}{=} \PY{n}{np}\PY{o}{.}\PY{n}{exp}\PY{p}{(}\PY{o}{\PYZhy{}}\PY{l+m+mi}{1}\PY{n}{j}\PY{o}{*}\PY{l+m+mi}{4}\PY{o}{*}\PY{n}{pi}\PY{o}{*}\PY{n}{n2}\PY{o}{/}\PY{l+m+mi}{3}\PY{p}{)} \PY{o}{*} \PY{p}{(}\PY{n}{h}\PY{p}{[}\PY{l+m+mi}{0}\PY{p}{,}\PY{n}{n2}\PY{p}{]} \PY{o}{+} \PY{n}{h}\PY{p}{[}\PY{l+m+mi}{1}\PY{p}{,}\PY{n}{n2}\PY{p}{]} \PY{o}{*} \PY{n}{np}\PY{o}{.}\PY{n}{exp}\PY{p}{(}\PY{o}{\PYZhy{}}\PY{l+m+mi}{1}\PY{n}{j}\PY{o}{*}\PY{l+m+mi}{2}\PY{o}{*}\PY{n}{pi}\PY{o}{/}\PY{l+m+mi}{3}\PY{p}{)} \PY{o}{+} \PY{n}{h}\PY{p}{[}\PY{l+m+mi}{2}\PY{p}{,}\PY{n}{n2}\PY{p}{]} \PY{o}{*} \PY{n}{np}\PY{o}{.}\PY{n}{exp}\PY{p}{(}\PY{o}{\PYZhy{}}\PY{l+m+mi}{1}\PY{n}{j}\PY{o}{*}\PY{l+m+mi}{4}\PY{o}{*}\PY{n}{pi}\PY{o}{/}\PY{l+m+mi}{3}\PY{p}{)}\PY{p}{)}
\PY{n+nb}{print}\PY{p}{(}\PY{l+s+s2}{\PYZdq{}}\PY{l+s+s2}{H(2,2) =}\PY{l+s+s2}{\PYZdq{}}\PY{p}{,}\PY{n+nb}{sum}\PY{p}{)}
        
\end{Verbatim}
\end{tcolorbox}

    \begin{Verbatim}[commandchars=\\\{\}]
H(1,1) = 0j
H(2,2) = 0j
    \end{Verbatim}

    \hypertarget{cross-term-calculation-example}{%
\subsection{Cross Term calculation
example:}\label{cross-term-calculation-example}}

    \begin{tcolorbox}[breakable, size=fbox, boxrule=1pt, pad at break*=1mm,colback=cellbackground, colframe=cellborder]
\prompt{In}{incolor}{17}{\boxspacing}
\begin{Verbatim}[commandchars=\\\{\}]
\PY{p}{(}\PY{l+m+mf}{0.25} \PY{o}{+} \PY{l+m+mi}{1}\PY{o}{*}\PY{n}{np}\PY{o}{.}\PY{n}{exp}\PY{p}{(}\PY{o}{\PYZhy{}}\PY{l+m+mi}{1}\PY{n}{j}\PY{o}{*}\PY{l+m+mi}{2}\PY{o}{*}\PY{n}{pi}\PY{o}{/}\PY{l+m+mi}{3}\PY{p}{)} \PY{o}{+} \PY{l+m+mf}{0.25}\PY{o}{*}\PY{n}{np}\PY{o}{.}\PY{n}{exp}\PY{p}{(}\PY{o}{\PYZhy{}}\PY{l+m+mi}{1}\PY{n}{j}\PY{o}{*}\PY{l+m+mi}{4}\PY{o}{*}\PY{n}{pi}\PY{o}{/}\PY{l+m+mi}{3}\PY{p}{)}\PY{p}{)}
\end{Verbatim}
\end{tcolorbox}

            \begin{tcolorbox}[breakable, size=fbox, boxrule=.5pt, pad at break*=1mm, opacityfill=0]
\prompt{Out}{outcolor}{17}{\boxspacing}
\begin{Verbatim}[commandchars=\\\{\}]
(-0.3749999999999999-0.6495190528383291j)
\end{Verbatim}
\end{tcolorbox}
        
    \begin{tcolorbox}[breakable, size=fbox, boxrule=1pt, pad at break*=1mm,colback=cellbackground, colframe=cellborder]
\prompt{In}{incolor}{18}{\boxspacing}
\begin{Verbatim}[commandchars=\\\{\}]
\PY{n}{fft2}\PY{p}{(}\PY{n}{H}\PY{p}{)}
\end{Verbatim}
\end{tcolorbox}

            \begin{tcolorbox}[breakable, size=fbox, boxrule=.5pt, pad at break*=1mm, opacityfill=0]
\prompt{Out}{outcolor}{18}{\boxspacing}
\begin{Verbatim}[commandchars=\\\{\}]
array([[ 1.5  -0.j        , -0.375-0.64951905j, -0.375+0.64951905j],
       [-0.375-0.64951905j,  0.   +0.j        ,  0.   -0.j        ],
       [-0.375+0.64951905j,  0.   +0.j        ,  0.   -0.j        ]])
\end{Verbatim}
\end{tcolorbox}
        
    \hypertarget{as-expected-we-again-obtained-an-low-pass-convolution-filter.-higher-frequency-terms-at-the-edges-of-the-filter-are-zero-whereas-center-terms-have-higher-magnitude.-this-is-totally-expected-since-dft-is-the-sampled-version-of-dtft-where-w-frac2pi-kn.}{%
\subsection{\texorpdfstring{As expected, we again obtained an low pass
convolution filter. Higher frequency terms ``at the edges'' of the
filter are zero, whereas center terms have higher magnitude. This is
totally expected, since DFT is the sampled version of DTFT where
\(w = \frac{2\pi k}{N}\).}{As expected, we again obtained an low pass convolution filter. Higher frequency terms ``at the edges'' of the filter are zero, whereas center terms have higher magnitude. This is totally expected, since DFT is the sampled version of DTFT where w = \textbackslash frac\{2\textbackslash pi k\}\{N\}.}}\label{as-expected-we-again-obtained-an-low-pass-convolution-filter.-higher-frequency-terms-at-the-edges-of-the-filter-are-zero-whereas-center-terms-have-higher-magnitude.-this-is-totally-expected-since-dft-is-the-sampled-version-of-dtft-where-w-frac2pi-kn.}}

    \hypertarget{q1-e}{%
\section{Q1 e}\label{q1-e}}

    \begin{tcolorbox}[breakable, size=fbox, boxrule=1pt, pad at break*=1mm,colback=cellbackground, colframe=cellborder]
\prompt{In}{incolor}{19}{\boxspacing}
\begin{Verbatim}[commandchars=\\\{\}]
\PY{k}{def} \PY{n+nf}{linear\PYZus{}conv\PYZus{}mat}\PY{p}{(}\PY{n}{h}\PY{p}{:}\PY{n}{np}\PY{o}{.}\PY{n}{array}\PY{p}{,}\PY{n}{output\PYZus{}size}\PY{p}{)}\PY{p}{:}
    \PY{n}{L} \PY{o}{=} \PY{n}{h}\PY{o}{.}\PY{n}{size}
    \PY{n}{h\PYZus{}ex} \PY{o}{=} \PY{n}{np}\PY{o}{.}\PY{n}{zeros}\PY{p}{(}\PY{n}{output\PYZus{}size}\PY{p}{)}
    \PY{n}{h\PYZus{}ex}\PY{p}{[}\PY{l+m+mi}{0}\PY{p}{:}\PY{n}{L}\PY{p}{]} \PY{o}{=} \PY{n}{h}
    \PY{n}{first\PYZus{}row} \PY{o}{=} \PY{n}{np}\PY{o}{.}\PY{n}{roll}\PY{p}{(}\PY{n}{np}\PY{o}{.}\PY{n}{flip}\PY{p}{(}\PY{n}{h\PYZus{}ex}\PY{p}{)}\PY{p}{,}\PY{l+m+mi}{1}\PY{p}{)}
    \PY{k}{return} \PY{n}{toeplitz}\PY{p}{(}\PY{n}{h\PYZus{}ex}\PY{o}{.}\PY{n}{T}\PY{p}{,}\PY{n}{first\PYZus{}row}\PY{p}{[}\PY{l+m+mi}{0}\PY{p}{:}\PY{p}{(}\PY{n}{output\PYZus{}size}\PY{o}{\PYZhy{}}\PY{n}{L}\PY{o}{+}\PY{l+m+mi}{1}\PY{p}{)}\PY{p}{]}\PY{p}{)} \PY{c+c1}{\PYZsh{} H + X \PYZhy{} 1 = output\PYZus{}size}
    
\PY{k}{def} \PY{n+nf}{linear\PYZus{}conv2\PYZus{}mat}\PY{p}{(}\PY{n}{h}\PY{p}{:}\PY{n}{np}\PY{o}{.}\PY{n}{ndarray}\PY{p}{,} \PY{n}{image\PYZus{}shape}\PY{p}{)}\PY{p}{:}
    \PY{n}{L1}\PY{p}{,}\PY{n}{L2} \PY{o}{=} \PY{n}{image\PYZus{}shape}
    \PY{n}{N1}\PY{p}{,}\PY{n}{N2} \PY{o}{=} \PY{n}{h}\PY{o}{.}\PY{n}{shape}
    \PY{n}{H} \PY{o}{=} \PY{n}{np}\PY{o}{.}\PY{n}{empty}\PY{p}{(}\PY{p}{(}\PY{p}{(}\PY{n}{N1}\PY{o}{+}\PY{n}{L1}\PY{o}{\PYZhy{}}\PY{l+m+mi}{1}\PY{p}{)}\PY{o}{*}\PY{p}{(}\PY{n}{N2}\PY{o}{+}\PY{n}{L2}\PY{o}{\PYZhy{}}\PY{l+m+mi}{1}\PY{p}{)}\PY{p}{,}\PY{l+m+mi}{0}\PY{p}{)}\PY{p}{)}
    \PY{n}{image\PYZus{}size} \PY{o}{=} \PY{n}{L1}\PY{o}{*}\PY{n}{L2}
    \PY{k}{for} \PY{n}{i} \PY{o+ow}{in} \PY{n+nb}{range}\PY{p}{(}\PY{n}{image\PYZus{}size}\PY{p}{)}\PY{p}{:}
        \PY{n}{row} \PY{o}{=} \PY{n}{np}\PY{o}{.}\PY{n}{mod}\PY{p}{(}\PY{n}{i}\PY{p}{,}\PY{n}{L1}\PY{p}{)}
        \PY{n}{col} \PY{o}{=} \PY{n}{i}\PY{o}{/}\PY{o}{/}\PY{n}{L1}
        \PY{n}{basis\PYZus{}vec} \PY{o}{=} \PY{n}{np}\PY{o}{.}\PY{n}{zeros}\PY{p}{(}\PY{p}{(}\PY{n}{L1}\PY{p}{,}\PY{n}{L2}\PY{p}{)}\PY{p}{)}
        \PY{n}{basis\PYZus{}vec}\PY{p}{[}\PY{n}{row}\PY{p}{,}\PY{n}{col}\PY{p}{]} \PY{o}{=} \PY{l+m+mi}{1}
        \PY{n}{basis\PYZus{}vec\PYZus{}output} \PY{o}{=} \PY{n}{convolve2d}\PY{p}{(}\PY{n}{h}\PY{p}{,}\PY{n}{basis\PYZus{}vec}\PY{p}{)}
        \PY{n}{H} \PY{o}{=} \PY{n}{np}\PY{o}{.}\PY{n}{column\PYZus{}stack}\PY{p}{(}\PY{p}{(}\PY{n}{H}\PY{p}{,} \PY{n}{basis\PYZus{}vec\PYZus{}output}\PY{o}{.}\PY{n}{flatten}\PY{p}{(}\PY{p}{)}\PY{p}{)}\PY{p}{)}
    \PY{k}{return} \PY{n}{H} 
\end{Verbatim}
\end{tcolorbox}

    \begin{tcolorbox}[breakable, size=fbox, boxrule=1pt, pad at break*=1mm,colback=cellbackground, colframe=cellborder]
\prompt{In}{incolor}{20}{\boxspacing}
\begin{Verbatim}[commandchars=\\\{\}]
\PY{k}{for} \PY{n}{\PYZus{}} \PY{o+ow}{in} \PY{n+nb}{range}\PY{p}{(}\PY{l+m+mi}{50}\PY{p}{)}\PY{p}{:}
    \PY{n}{N1} \PY{o}{=} \PY{n}{np}\PY{o}{.}\PY{n}{random}\PY{o}{.}\PY{n}{randint}\PY{p}{(}\PY{l+m+mi}{0}\PY{p}{,}\PY{l+m+mi}{20}\PY{p}{)}
    \PY{n}{N2} \PY{o}{=} \PY{n}{np}\PY{o}{.}\PY{n}{random}\PY{o}{.}\PY{n}{randint}\PY{p}{(}\PY{l+m+mi}{0}\PY{p}{,}\PY{l+m+mi}{20}\PY{p}{)}
    \PY{n}{L1} \PY{o}{=} \PY{n}{np}\PY{o}{.}\PY{n}{random}\PY{o}{.}\PY{n}{randint}\PY{p}{(}\PY{l+m+mi}{0}\PY{p}{,}\PY{l+m+mi}{20}\PY{p}{)}
    \PY{n}{L2} \PY{o}{=} \PY{n}{np}\PY{o}{.}\PY{n}{random}\PY{o}{.}\PY{n}{randint}\PY{p}{(}\PY{l+m+mi}{0}\PY{p}{,}\PY{l+m+mi}{20}\PY{p}{)}
    \PY{n}{H} \PY{o}{=} \PY{n}{np}\PY{o}{.}\PY{n}{random}\PY{o}{.}\PY{n}{randint}\PY{p}{(}\PY{l+m+mi}{0}\PY{p}{,}\PY{l+m+mi}{100}\PY{p}{,}\PY{p}{(}\PY{n}{N1}\PY{p}{,}\PY{n}{N2}\PY{p}{)}\PY{p}{)}
    \PY{n}{x} \PY{o}{=} \PY{n}{np}\PY{o}{.}\PY{n}{random}\PY{o}{.}\PY{n}{randint}\PY{p}{(}\PY{l+m+mi}{0}\PY{p}{,}\PY{l+m+mi}{100}\PY{p}{,}\PY{p}{(}\PY{n}{L1}\PY{p}{,}\PY{n}{L2}\PY{p}{)}\PY{p}{)}
    \PY{n}{H\PYZus{}mat} \PY{o}{=} \PY{n}{linear\PYZus{}conv2\PYZus{}mat}\PY{p}{(}\PY{n}{h}\PY{o}{=}\PY{n}{H}\PY{p}{,}\PY{n}{image\PYZus{}shape}\PY{o}{=}\PY{n}{x}\PY{o}{.}\PY{n}{shape}\PY{p}{)}
    \PY{n}{y\PYZus{}prime} \PY{o}{=} \PY{n}{convolve2d}\PY{p}{(}\PY{n}{H}\PY{p}{,}\PY{n}{x}\PY{p}{)}
    \PY{n}{y\PYZus{}prime\PYZus{}vec} \PY{o}{=} \PY{n}{H\PYZus{}mat}\PY{o}{.}\PY{n}{dot}\PY{p}{(}\PY{n}{x}\PY{o}{.}\PY{n}{flatten}\PY{p}{(}\PY{p}{)}\PY{p}{)}
    \PY{n}{y\PYZus{}prime\PYZus{}back} \PY{o}{=} \PY{n}{np}\PY{o}{.}\PY{n}{reshape}\PY{p}{(}\PY{n}{y\PYZus{}prime\PYZus{}vec}\PY{p}{,}\PY{n}{y\PYZus{}prime}\PY{o}{.}\PY{n}{shape}\PY{p}{,}\PY{n}{order}\PY{o}{=}\PY{l+s+s2}{\PYZdq{}}\PY{l+s+s2}{F}\PY{l+s+s2}{\PYZdq{}}\PY{p}{)}

    \PY{k}{assert} \PY{n}{np}\PY{o}{.}\PY{n}{isclose}\PY{p}{(}\PY{n}{np}\PY{o}{.}\PY{n}{sum}\PY{p}{(}\PY{n}{y\PYZus{}prime}\PY{o}{\PYZhy{}}\PY{n}{y\PYZus{}prime\PYZus{}back}\PY{p}{)}\PY{p}{,}\PY{l+m+mi}{0}\PY{p}{)} \PY{c+c1}{\PYZsh{} Check if they are the same }
\PY{n+nb}{print}\PY{p}{(}\PY{l+s+s2}{\PYZdq{}}\PY{l+s+s2}{linear\PYZus{}conv2\PYZus{}mat works}\PY{l+s+s2}{\PYZdq{}}\PY{p}{)}
\end{Verbatim}
\end{tcolorbox}

    \begin{Verbatim}[commandchars=\\\{\}]
linear\_conv2\_mat works
    \end{Verbatim}

    \begin{tcolorbox}[breakable, size=fbox, boxrule=1pt, pad at break*=1mm,colback=cellbackground, colframe=cellborder]
\prompt{In}{incolor}{21}{\boxspacing}
\begin{Verbatim}[commandchars=\\\{\}]
\PY{n}{H} \PY{o}{=} \PY{n}{np}\PY{o}{.}\PY{n}{random}\PY{o}{.}\PY{n}{randint}\PY{p}{(}\PY{l+m+mi}{5}\PY{p}{,}\PY{l+m+mi}{10}\PY{p}{,}\PY{p}{(}\PY{l+m+mi}{3}\PY{p}{,}\PY{l+m+mi}{3}\PY{p}{)}\PY{p}{)}
\PY{n}{x} \PY{o}{=} \PY{n}{np}\PY{o}{.}\PY{n}{random}\PY{o}{.}\PY{n}{randint}\PY{p}{(}\PY{l+m+mi}{5}\PY{p}{,}\PY{l+m+mi}{10}\PY{p}{,}\PY{p}{(}\PY{l+m+mi}{7}\PY{p}{,}\PY{l+m+mi}{7}\PY{p}{)}\PY{p}{)}
\PY{n}{H\PYZus{}mat} \PY{o}{=} \PY{n}{linear\PYZus{}conv2\PYZus{}mat}\PY{p}{(}\PY{n}{h}\PY{o}{=}\PY{n}{H}\PY{p}{,}\PY{n}{image\PYZus{}shape}\PY{o}{=}\PY{n}{x}\PY{o}{.}\PY{n}{shape}\PY{p}{)}
\PY{n}{plt}\PY{o}{.}\PY{n}{figure}\PY{p}{(}\PY{n}{figsize}\PY{o}{=}\PY{p}{(}\PY{l+m+mi}{18}\PY{p}{,}\PY{l+m+mi}{9}\PY{p}{)}\PY{p}{)}
\PY{n}{plt}\PY{o}{.}\PY{n}{imshow}\PY{p}{(}\PY{n}{H\PYZus{}mat}\PY{p}{,}\PY{n}{cmap}\PY{o}{=}\PY{l+s+s2}{\PYZdq{}}\PY{l+s+s2}{jet}\PY{l+s+s2}{\PYZdq{}}\PY{p}{)}
\end{Verbatim}
\end{tcolorbox}

            \begin{tcolorbox}[breakable, size=fbox, boxrule=.5pt, pad at break*=1mm, opacityfill=0]
\prompt{Out}{outcolor}{21}{\boxspacing}
\begin{Verbatim}[commandchars=\\\{\}]
<matplotlib.image.AxesImage at 0x25fe7fec550>
\end{Verbatim}
\end{tcolorbox}
        
    \begin{center}
    \adjustimage{max size={0.9\linewidth}{0.9\paperheight}}{HW_1_files/HW_1_20_1.png}
    \end{center}
    { \hspace*{\fill} \\}
    
    \hypertarget{q1f}{%
\section{Q1f}\label{q1f}}

    Let the \(i^{th}\) column of the image to convolved is called \(x_i\).
Then, the matrix product \(Cx\) can be considered as ``convolution with
matrices \(s_i\) and the vectors \(x_i\)''. Let the output image be y
and the \(i^{th}\) column of it is called \(y_i\). Then:

\[ y_i = \sum\limits_{k=0}^{L_2}s_{[i-k]}x_k \]

where \(sk\) corresponds to the matrix-vector product with indices
higher than \(P_2\) and negative indices of s correspond to zero
multiplication. Since \(s\) matrix is a Toeplitz matrix, this product is
equal to a convolution with a vector which circulates in the columns of
this Toeplitz matrix, \emph{i.e.} the columns of the filter kernel.

    \begin{tcolorbox}[breakable, size=fbox, boxrule=1pt, pad at break*=1mm,colback=cellbackground, colframe=cellborder]
\prompt{In}{incolor}{22}{\boxspacing}
\begin{Verbatim}[commandchars=\\\{\}]
\PY{k}{def} \PY{n+nf}{conv2\PYZus{}by\PYZus{}fft}\PY{p}{(}\PY{n}{x}\PY{p}{:}\PY{n}{np}\PY{o}{.}\PY{n}{ndarray}\PY{p}{,} \PY{n}{h}\PY{p}{:}\PY{n}{np}\PY{o}{.}\PY{n}{ndarray}\PY{p}{)}\PY{p}{:}
    \PY{k}{if} \PY{n}{h}\PY{o}{.}\PY{n}{shape}\PY{p}{[}\PY{l+m+mi}{0}\PY{p}{]} \PY{o}{\PYZgt{}} \PY{n}{x}\PY{o}{.}\PY{n}{shape}\PY{p}{[}\PY{l+m+mi}{0}\PY{p}{]} \PY{o+ow}{and} \PY{n}{h}\PY{o}{.}\PY{n}{shape}\PY{p}{[}\PY{l+m+mi}{1}\PY{p}{]} \PY{o}{\PYZgt{}} \PY{n}{x}\PY{o}{.}\PY{n}{shape}\PY{p}{[}\PY{l+m+mi}{1}\PY{p}{]}\PY{p}{:}
        \PY{n}{x}\PY{p}{,}\PY{n}{h} \PY{o}{=} \PY{n}{h}\PY{p}{,}\PY{n}{x}
    \PY{n}{L1}\PY{p}{,}\PY{n}{L2} \PY{o}{=} \PY{n}{x}\PY{o}{.}\PY{n}{shape}
    \PY{n}{P1}\PY{p}{,}\PY{n}{P2} \PY{o}{=} \PY{n}{h}\PY{o}{.}\PY{n}{shape}
    \PY{n}{rows} \PY{o}{=} \PY{n}{L1}\PY{o}{+}\PY{n}{P1}\PY{o}{\PYZhy{}}\PY{l+m+mi}{1}
    \PY{n}{cols} \PY{o}{=} \PY{n}{L2}\PY{o}{+}\PY{n}{P2}\PY{o}{\PYZhy{}}\PY{l+m+mi}{1}
    \PY{n}{Y} \PY{o}{=} \PY{n}{np}\PY{o}{.}\PY{n}{empty}\PY{p}{(}\PY{p}{(}\PY{n}{rows}\PY{p}{,}\PY{n}{cols}\PY{p}{)}\PY{p}{)}
    \PY{k}{for} \PY{n}{i} \PY{o+ow}{in} \PY{n+nb}{range}\PY{p}{(}\PY{n}{cols}\PY{p}{)}\PY{p}{:}
        \PY{n}{output\PYZus{}col} \PY{o}{=} \PY{n}{np}\PY{o}{.}\PY{n}{zeros}\PY{p}{(}\PY{p}{(}\PY{n}{rows}\PY{p}{,}\PY{p}{)}\PY{p}{)}
        \PY{k}{for} \PY{n}{p} \PY{o+ow}{in} \PY{n+nb}{range}\PY{p}{(}\PY{n}{L2}\PY{p}{)}\PY{p}{:} \PY{c+c1}{\PYZsh{} travel through image columns}
            \PY{k}{if} \PY{n}{i}\PY{o}{\PYZhy{}}\PY{n}{p} \PY{o}{\PYZgt{}}\PY{o}{=} \PY{l+m+mi}{0} \PY{o+ow}{and} \PY{n}{i}\PY{o}{\PYZhy{}}\PY{n}{p} \PY{o}{\PYZlt{}} \PY{n}{P2}\PY{p}{:}
                \PY{n}{product} \PY{o}{=} \PY{n}{np}\PY{o}{.}\PY{n}{multiply}\PY{p}{(}\PY{n}{fft}\PY{p}{(}\PY{n}{x}\PY{p}{[}\PY{p}{:}\PY{p}{,}\PY{n}{p}\PY{p}{]}\PY{p}{,}\PY{n}{rows}\PY{p}{)}\PY{p}{,}\PY{n}{fft}\PY{p}{(}\PY{n}{h}\PY{p}{[}\PY{p}{:}\PY{p}{,}\PY{n}{i}\PY{o}{\PYZhy{}}\PY{n}{p}\PY{p}{]}\PY{p}{,}\PY{n}{rows}\PY{p}{)}\PY{p}{)} \PY{c+c1}{\PYZsh{} splitted lines for debugging}
                \PY{n}{output\PYZus{}col} \PY{o}{+}\PY{o}{=} \PY{n}{np}\PY{o}{.}\PY{n}{real}\PY{p}{(}\PY{n}{ifft}\PY{p}{(}\PY{n}{product}\PY{p}{)}\PY{p}{)}
                \PY{n}{Y}\PY{p}{[}\PY{p}{:}\PY{p}{,}\PY{n}{i}\PY{p}{]} \PY{o}{=} \PY{n}{output\PYZus{}col}
    \PY{k}{return} \PY{n}{Y}
        
\end{Verbatim}
\end{tcolorbox}

    \hypertarget{test-the-function}{%
\subsection{Test the function}\label{test-the-function}}

    \begin{tcolorbox}[breakable, size=fbox, boxrule=1pt, pad at break*=1mm,colback=cellbackground, colframe=cellborder]
\prompt{In}{incolor}{23}{\boxspacing}
\begin{Verbatim}[commandchars=\\\{\}]
\PY{k}{for} \PY{n}{\PYZus{}} \PY{o+ow}{in} \PY{n+nb}{range}\PY{p}{(}\PY{l+m+mi}{1500}\PY{p}{)}\PY{p}{:}
    \PY{n}{L1} \PY{o}{=} \PY{n}{np}\PY{o}{.}\PY{n}{random}\PY{o}{.}\PY{n}{randint}\PY{p}{(}\PY{l+m+mi}{3}\PY{p}{,}\PY{l+m+mi}{15}\PY{p}{)}
    \PY{n}{L2} \PY{o}{=} \PY{n}{np}\PY{o}{.}\PY{n}{random}\PY{o}{.}\PY{n}{randint}\PY{p}{(}\PY{l+m+mi}{3}\PY{p}{,}\PY{l+m+mi}{15}\PY{p}{)}
    \PY{n}{N1} \PY{o}{=} \PY{n}{np}\PY{o}{.}\PY{n}{random}\PY{o}{.}\PY{n}{randint}\PY{p}{(}\PY{l+m+mi}{1}\PY{p}{,}\PY{n}{L1}\PY{o}{\PYZhy{}}\PY{l+m+mi}{1}\PY{p}{)}
    \PY{n}{N2} \PY{o}{=} \PY{n}{np}\PY{o}{.}\PY{n}{random}\PY{o}{.}\PY{n}{randint}\PY{p}{(}\PY{l+m+mi}{1}\PY{p}{,}\PY{n}{L2}\PY{o}{\PYZhy{}}\PY{l+m+mi}{1}\PY{p}{)}
    \PY{n}{H} \PY{o}{=} \PY{n}{np}\PY{o}{.}\PY{n}{random}\PY{o}{.}\PY{n}{randint}\PY{p}{(}\PY{l+m+mi}{0}\PY{p}{,}\PY{l+m+mi}{100}\PY{p}{,}\PY{p}{(}\PY{n}{N1}\PY{p}{,}\PY{n}{N2}\PY{p}{)}\PY{p}{)}
    \PY{n}{x} \PY{o}{=} \PY{n}{np}\PY{o}{.}\PY{n}{random}\PY{o}{.}\PY{n}{randint}\PY{p}{(}\PY{l+m+mi}{0}\PY{p}{,}\PY{l+m+mi}{100}\PY{p}{,}\PY{p}{(}\PY{n}{L1}\PY{p}{,}\PY{n}{L2}\PY{p}{)}\PY{p}{)}
    \PY{n}{y} \PY{o}{=} \PY{n}{conv2\PYZus{}by\PYZus{}fft}\PY{p}{(}\PY{n}{x}\PY{p}{,}\PY{n}{H}\PY{p}{)}
    \PY{n}{y\PYZus{}prime} \PY{o}{=} \PY{n}{convolve2d}\PY{p}{(}\PY{n}{x}\PY{p}{,}\PY{n}{H}\PY{p}{)}
    \PY{k}{assert} \PY{n}{np}\PY{o}{.}\PY{n}{isclose}\PY{p}{(}\PY{n}{np}\PY{o}{.}\PY{n}{sum}\PY{p}{(}\PY{n}{y\PYZus{}prime}\PY{o}{\PYZhy{}}\PY{n}{y}\PY{p}{)}\PY{p}{,}\PY{l+m+mi}{0}\PY{p}{)} \PY{c+c1}{\PYZsh{} Check if they are the same }
\PY{n+nb}{print}\PY{p}{(}\PY{l+s+s2}{\PYZdq{}}\PY{l+s+s2}{conv2\PYZus{}by\PYZus{}fft works}\PY{l+s+s2}{\PYZdq{}}\PY{p}{)}
\end{Verbatim}
\end{tcolorbox}

    \begin{Verbatim}[commandchars=\\\{\}]
conv2\_by\_fft works
    \end{Verbatim}

    \hypertarget{q1g}{%
\section{Q1g}\label{q1g}}

    \begin{tcolorbox}[breakable, size=fbox, boxrule=1pt, pad at break*=1mm,colback=cellbackground, colframe=cellborder]
\prompt{In}{incolor}{24}{\boxspacing}
\begin{Verbatim}[commandchars=\\\{\}]
\PY{n}{image} \PY{o}{=} \PY{n}{plt}\PY{o}{.}\PY{n}{imread}\PY{p}{(}\PY{l+s+s2}{\PYZdq{}}\PY{l+s+s2}{mandrill.bmp}\PY{l+s+s2}{\PYZdq{}}\PY{p}{)}
\PY{n}{h} \PY{o}{=} \PY{n}{np}\PY{o}{.}\PY{n}{array}\PY{p}{(}\PY{p}{[}\PY{p}{[}\PY{l+m+mi}{0}\PY{p}{,}\PY{o}{.}\PY{l+m+mi}{25}\PY{p}{,}\PY{l+m+mi}{0}\PY{p}{]}\PY{p}{,}\PY{p}{[}\PY{o}{.}\PY{l+m+mi}{25}\PY{p}{,}\PY{o}{.}\PY{l+m+mi}{5}\PY{p}{,}\PY{o}{.}\PY{l+m+mi}{25}\PY{p}{]}\PY{p}{,}\PY{p}{[}\PY{l+m+mi}{0}\PY{p}{,}\PY{o}{.}\PY{l+m+mi}{25}\PY{p}{,}\PY{l+m+mi}{0}\PY{p}{]}\PY{p}{]}\PY{p}{)}
\PY{n}{blurred\PYZus{}image} \PY{o}{=} \PY{n}{conv2\PYZus{}by\PYZus{}fft}\PY{p}{(}\PY{n}{h}\PY{p}{,}\PY{n}{image}\PY{p}{)}
\PY{n}{plt}\PY{o}{.}\PY{n}{figure}\PY{p}{(}\PY{n}{figsize}\PY{o}{=}\PY{p}{(}\PY{l+m+mi}{10}\PY{p}{,}\PY{l+m+mi}{40}\PY{p}{)}\PY{p}{)}
\PY{n}{plt}\PY{o}{.}\PY{n}{subplot}\PY{p}{(}\PY{l+m+mi}{1}\PY{p}{,}\PY{l+m+mi}{3}\PY{p}{,}\PY{l+m+mi}{1}\PY{p}{)}
\PY{n}{plt}\PY{o}{.}\PY{n}{imshow}\PY{p}{(}\PY{n}{image}\PY{p}{)}
\PY{n}{plt}\PY{o}{.}\PY{n}{title}\PY{p}{(}\PY{l+s+s2}{\PYZdq{}}\PY{l+s+s2}{Original image}\PY{l+s+s2}{\PYZdq{}}\PY{p}{,}\PY{n}{fontsize}\PY{o}{=}\PY{l+m+mi}{20}\PY{p}{)}
\PY{n}{plt}\PY{o}{.}\PY{n}{subplot}\PY{p}{(}\PY{l+m+mi}{1}\PY{p}{,}\PY{l+m+mi}{3}\PY{p}{,}\PY{l+m+mi}{2}\PY{p}{)}
\PY{n}{plt}\PY{o}{.}\PY{n}{imshow}\PY{p}{(}\PY{n}{blurred\PYZus{}image}\PY{p}{,}\PY{n}{vmin}\PY{o}{=}\PY{n}{np}\PY{o}{.}\PY{n}{min}\PY{p}{(}\PY{n}{image}\PY{p}{)}\PY{p}{,} \PY{n}{vmax}\PY{o}{=}\PY{n}{np}\PY{o}{.}\PY{n}{max}\PY{p}{(}\PY{n}{image}\PY{p}{)}\PY{p}{)}
\PY{n}{blurred\PYZus{}image\PYZus{}normalized} \PY{o}{=} \PY{n}{conv2\PYZus{}by\PYZus{}fft}\PY{p}{(}\PY{n}{h}\PY{o}{/}\PY{n}{np}\PY{o}{.}\PY{n}{sum}\PY{p}{(}\PY{n}{h}\PY{p}{)}\PY{p}{,}\PY{n}{image}\PY{p}{)}
\PY{n}{plt}\PY{o}{.}\PY{n}{title}\PY{p}{(}\PY{l+s+s2}{\PYZdq{}}\PY{l+s+s2}{Blurred Image}\PY{l+s+s2}{\PYZdq{}}\PY{p}{,}\PY{n}{fontsize}\PY{o}{=}\PY{l+m+mi}{20}\PY{p}{)}
\PY{n}{plt}\PY{o}{.}\PY{n}{subplot}\PY{p}{(}\PY{l+m+mi}{1}\PY{p}{,}\PY{l+m+mi}{3}\PY{p}{,}\PY{l+m+mi}{3}\PY{p}{)}
\PY{n}{plt}\PY{o}{.}\PY{n}{imshow}\PY{p}{(}\PY{n}{blurred\PYZus{}image\PYZus{}normalized}\PY{p}{,}\PY{n}{vmin}\PY{o}{=}\PY{n}{np}\PY{o}{.}\PY{n}{min}\PY{p}{(}\PY{n}{image}\PY{p}{)}\PY{p}{,} \PY{n}{vmax}\PY{o}{=}\PY{n}{np}\PY{o}{.}\PY{n}{max}\PY{p}{(}\PY{n}{image}\PY{p}{)}\PY{p}{)}
\PY{n}{plt}\PY{o}{.}\PY{n}{title}\PY{p}{(}\PY{l+s+s2}{\PYZdq{}}\PY{l+s+s2}{Blurred image with }\PY{l+s+se}{\PYZbs{}n}\PY{l+s+s2}{ normalized kernel}\PY{l+s+s2}{\PYZdq{}}\PY{p}{,}\PY{n}{fontsize}\PY{o}{=}\PY{l+m+mi}{20}\PY{p}{)}
\end{Verbatim}
\end{tcolorbox}

            \begin{tcolorbox}[breakable, size=fbox, boxrule=.5pt, pad at break*=1mm, opacityfill=0]
\prompt{Out}{outcolor}{24}{\boxspacing}
\begin{Verbatim}[commandchars=\\\{\}]
Text(0.5, 1.0, 'Blurred image with \textbackslash{}n normalized kernel')
\end{Verbatim}
\end{tcolorbox}
        
    \begin{center}
    \adjustimage{max size={0.9\linewidth}{0.9\paperheight}}{HW_1_files/HW_1_27_1.png}
    \end{center}
    { \hspace*{\fill} \\}
    
    \hypertarget{in-the-last-image-we-observe-that-the-hairy-part-of-the-cheeks-of-the-mandrill-got-blurred-and-is-not-distinctive-as-it-is-in-the-original-image-anymore.}{%
\subsection{In the last image, we observe that the hairy part of the
cheeks of the mandrill got blurred and is not distinctive as it is in
the original image
anymore.}\label{in-the-last-image-we-observe-that-the-hairy-part-of-the-cheeks-of-the-mandrill-got-blurred-and-is-not-distinctive-as-it-is-in-the-original-image-anymore.}}

    \hypertarget{q2}{%
\section{Q2}\label{q2}}

    \begin{tcolorbox}[breakable, size=fbox, boxrule=1pt, pad at break*=1mm,colback=cellbackground, colframe=cellborder]
\prompt{In}{incolor}{25}{\boxspacing}
\begin{Verbatim}[commandchars=\\\{\}]
\PY{n}{plt}\PY{o}{.}\PY{n}{figure}\PY{p}{(}\PY{n}{figsize}\PY{o}{=}\PY{p}{(}\PY{l+m+mi}{25}\PY{p}{,}\PY{l+m+mi}{10}\PY{p}{)}\PY{p}{)}

\PY{n}{plt}\PY{o}{.}\PY{n}{subplot}\PY{p}{(}\PY{l+m+mi}{1}\PY{p}{,}\PY{l+m+mi}{2}\PY{p}{,}\PY{l+m+mi}{1}\PY{p}{)}
\PY{n}{ax} \PY{o}{=} \PY{n}{plt}\PY{o}{.}\PY{n}{gca}\PY{p}{(}\PY{p}{)}
\PY{n}{ax}\PY{o}{.}\PY{n}{cla}\PY{p}{(}\PY{p}{)} \PY{c+c1}{\PYZsh{} clear things for fresh plot}
\PY{n}{circle1} \PY{o}{=} \PY{n}{plt}\PY{o}{.}\PY{n}{Circle}\PY{p}{(}\PY{p}{(}\PY{l+m+mi}{0}\PY{p}{,} \PY{l+m+mi}{0}\PY{p}{)}\PY{p}{,} \PY{o}{.}\PY{l+m+mi}{5}\PY{p}{,} \PY{n}{color}\PY{o}{=}\PY{l+s+s1}{\PYZsq{}}\PY{l+s+s1}{r}\PY{l+s+s1}{\PYZsq{}}\PY{p}{)}
\PY{n}{circle2} \PY{o}{=} \PY{n}{plt}\PY{o}{.}\PY{n}{Circle}\PY{p}{(}\PY{p}{(}\PY{l+m+mi}{1}\PY{p}{,} \PY{l+m+mi}{1}\PY{p}{)}\PY{p}{,} \PY{o}{.}\PY{l+m+mi}{5}\PY{p}{,} \PY{n}{color}\PY{o}{=}\PY{l+s+s1}{\PYZsq{}}\PY{l+s+s1}{g}\PY{l+s+s1}{\PYZsq{}}\PY{p}{)}
\PY{n}{circle3} \PY{o}{=} \PY{n}{plt}\PY{o}{.}\PY{n}{Circle}\PY{p}{(}\PY{p}{(}\PY{l+m+mi}{0}\PY{p}{,} \PY{l+m+mi}{1}\PY{p}{)}\PY{p}{,} \PY{o}{.}\PY{l+m+mi}{5}\PY{p}{,} \PY{n}{color}\PY{o}{=}\PY{l+s+s1}{\PYZsq{}}\PY{l+s+s1}{b}\PY{l+s+s1}{\PYZsq{}}\PY{p}{)}
\PY{n}{circle4} \PY{o}{=} \PY{n}{plt}\PY{o}{.}\PY{n}{Circle}\PY{p}{(}\PY{p}{(}\PY{l+m+mi}{1}\PY{p}{,} \PY{l+m+mi}{0}\PY{p}{)}\PY{p}{,} \PY{o}{.}\PY{l+m+mi}{5}\PY{p}{,} \PY{n}{color}\PY{o}{=}\PY{l+s+s1}{\PYZsq{}}\PY{l+s+s1}{magenta}\PY{l+s+s1}{\PYZsq{}}\PY{p}{)}
\PY{n}{circle5} \PY{o}{=} \PY{n}{plt}\PY{o}{.}\PY{n}{Circle}\PY{p}{(}\PY{p}{(}\PY{o}{\PYZhy{}}\PY{l+m+mi}{1}\PY{p}{,} \PY{l+m+mi}{0}\PY{p}{)}\PY{p}{,} \PY{o}{.}\PY{l+m+mi}{5}\PY{p}{,} \PY{n}{color}\PY{o}{=}\PY{l+s+s1}{\PYZsq{}}\PY{l+s+s1}{k}\PY{l+s+s1}{\PYZsq{}}\PY{p}{)}
\PY{n}{circle6} \PY{o}{=} \PY{n}{plt}\PY{o}{.}\PY{n}{Circle}\PY{p}{(}\PY{p}{(}\PY{o}{\PYZhy{}}\PY{l+m+mi}{1}\PY{p}{,} \PY{o}{\PYZhy{}}\PY{l+m+mi}{1}\PY{p}{)}\PY{p}{,} \PY{o}{.}\PY{l+m+mi}{5}\PY{p}{,} \PY{n}{color}\PY{o}{=}\PY{l+s+s1}{\PYZsq{}}\PY{l+s+s1}{g}\PY{l+s+s1}{\PYZsq{}}\PY{p}{)}
\PY{n}{circle7} \PY{o}{=} \PY{n}{plt}\PY{o}{.}\PY{n}{Circle}\PY{p}{(}\PY{p}{(}\PY{l+m+mi}{0}\PY{p}{,} \PY{o}{\PYZhy{}}\PY{l+m+mi}{1}\PY{p}{)}\PY{p}{,} \PY{o}{.}\PY{l+m+mi}{5}\PY{p}{,} \PY{n}{color}\PY{o}{=}\PY{l+s+s1}{\PYZsq{}}\PY{l+s+s1}{b}\PY{l+s+s1}{\PYZsq{}}\PY{p}{)}
\PY{n}{circle8} \PY{o}{=} \PY{n}{plt}\PY{o}{.}\PY{n}{Circle}\PY{p}{(}\PY{p}{(}\PY{l+m+mi}{1}\PY{p}{,} \PY{o}{\PYZhy{}}\PY{l+m+mi}{1}\PY{p}{)}\PY{p}{,} \PY{o}{.}\PY{l+m+mi}{5}\PY{p}{,} \PY{n}{color}\PY{o}{=}\PY{l+s+s1}{\PYZsq{}}\PY{l+s+s1}{k}\PY{l+s+s1}{\PYZsq{}}\PY{p}{)}
\PY{n}{circle9} \PY{o}{=} \PY{n}{plt}\PY{o}{.}\PY{n}{Circle}\PY{p}{(}\PY{p}{(}\PY{o}{\PYZhy{}}\PY{l+m+mi}{1}\PY{p}{,} \PY{l+m+mi}{1}\PY{p}{)}\PY{p}{,} \PY{o}{.}\PY{l+m+mi}{5}\PY{p}{,} \PY{n}{color}\PY{o}{=}\PY{l+s+s1}{\PYZsq{}}\PY{l+s+s1}{r}\PY{l+s+s1}{\PYZsq{}}\PY{p}{)}
\PY{c+c1}{\PYZsh{} change default range so that new circles will work}
\PY{n}{ax}\PY{o}{.}\PY{n}{set\PYZus{}xlim}\PY{p}{(}\PY{p}{(}\PY{o}{\PYZhy{}}\PY{l+m+mi}{4}\PY{p}{,} \PY{l+m+mi}{4}\PY{p}{)}\PY{p}{)}
\PY{n}{ax}\PY{o}{.}\PY{n}{set\PYZus{}ylim}\PY{p}{(}\PY{p}{(}\PY{o}{\PYZhy{}}\PY{l+m+mi}{4}\PY{p}{,} \PY{l+m+mi}{4}\PY{p}{)}\PY{p}{)}
\PY{n}{ax}\PY{o}{.}\PY{n}{add\PYZus{}patch}\PY{p}{(}\PY{n}{circle1}\PY{p}{)}
\PY{n}{ax}\PY{o}{.}\PY{n}{add\PYZus{}patch}\PY{p}{(}\PY{n}{circle2}\PY{p}{)}
\PY{n}{ax}\PY{o}{.}\PY{n}{add\PYZus{}patch}\PY{p}{(}\PY{n}{circle3}\PY{p}{)}
\PY{n}{ax}\PY{o}{.}\PY{n}{add\PYZus{}patch}\PY{p}{(}\PY{n}{circle4}\PY{p}{)}
\PY{n}{ax}\PY{o}{.}\PY{n}{add\PYZus{}patch}\PY{p}{(}\PY{n}{circle5}\PY{p}{)}
\PY{n}{ax}\PY{o}{.}\PY{n}{add\PYZus{}patch}\PY{p}{(}\PY{n}{circle6}\PY{p}{)}
\PY{n}{ax}\PY{o}{.}\PY{n}{add\PYZus{}patch}\PY{p}{(}\PY{n}{circle7}\PY{p}{)}
\PY{n}{ax}\PY{o}{.}\PY{n}{add\PYZus{}patch}\PY{p}{(}\PY{n}{circle8}\PY{p}{)}
\PY{n}{ax}\PY{o}{.}\PY{n}{add\PYZus{}patch}\PY{p}{(}\PY{n}{circle9}\PY{p}{)}
\PY{n}{plt}\PY{o}{.}\PY{n}{title}\PY{p}{(}\PY{l+s+s2}{\PYZdq{}}\PY{l+s+s2}{Rectangular Grid}\PY{l+s+s2}{\PYZdq{}}\PY{p}{,}\PY{n}{fontsize}\PY{o}{=}\PY{l+m+mi}{20}\PY{p}{)}
\PY{n}{plt}\PY{o}{.}\PY{n}{xlabel}\PY{p}{(}\PY{l+s+sa}{r}\PY{l+s+s1}{\PYZsq{}}\PY{l+s+s1}{\PYZdl{}f\PYZus{}1\PYZdl{}}\PY{l+s+s1}{\PYZsq{}}\PY{p}{,}\PY{n}{fontsize}\PY{o}{=}\PY{l+m+mi}{20}\PY{p}{)}
\PY{n}{plt}\PY{o}{.}\PY{n}{ylabel}\PY{p}{(}\PY{l+s+sa}{r}\PY{l+s+s1}{\PYZsq{}}\PY{l+s+s1}{\PYZdl{}f\PYZus{}2\PYZdl{}}\PY{l+s+s1}{\PYZsq{}}\PY{p}{,}\PY{n}{fontsize}\PY{o}{=}\PY{l+m+mi}{20}\PY{p}{)}


\PY{n}{plt}\PY{o}{.}\PY{n}{subplot}\PY{p}{(}\PY{l+m+mi}{1}\PY{p}{,}\PY{l+m+mi}{2}\PY{p}{,}\PY{l+m+mi}{2}\PY{p}{)}
\PY{n}{ax} \PY{o}{=} \PY{n}{plt}\PY{o}{.}\PY{n}{gca}\PY{p}{(}\PY{p}{)}
\PY{n}{ax}\PY{o}{.}\PY{n}{cla}\PY{p}{(}\PY{p}{)} \PY{c+c1}{\PYZsh{} clear things for fresh plot}
\PY{n}{circle1} \PY{o}{=} \PY{n}{plt}\PY{o}{.}\PY{n}{Circle}\PY{p}{(}\PY{p}{(}\PY{l+m+mi}{0}\PY{p}{,} \PY{l+m+mi}{0}\PY{p}{)}\PY{p}{,} \PY{o}{.}\PY{l+m+mi}{5}\PY{p}{,} \PY{n}{color}\PY{o}{=}\PY{l+s+s1}{\PYZsq{}}\PY{l+s+s1}{r}\PY{l+s+s1}{\PYZsq{}}\PY{p}{)}
\PY{n}{circle2} \PY{o}{=} \PY{n}{plt}\PY{o}{.}\PY{n}{Circle}\PY{p}{(}\PY{p}{(}\PY{o}{\PYZhy{}}\PY{l+m+mf}{0.5}\PY{p}{,} \PY{l+m+mf}{0.5}\PY{o}{*}\PY{n}{np}\PY{o}{.}\PY{n}{sqrt}\PY{p}{(}\PY{l+m+mi}{3}\PY{p}{)}\PY{p}{)}\PY{p}{,} \PY{o}{.}\PY{l+m+mi}{5}\PY{p}{,} \PY{n}{color}\PY{o}{=}\PY{l+s+s1}{\PYZsq{}}\PY{l+s+s1}{g}\PY{l+s+s1}{\PYZsq{}}\PY{p}{)}
\PY{n}{circle3} \PY{o}{=} \PY{n}{plt}\PY{o}{.}\PY{n}{Circle}\PY{p}{(}\PY{p}{(}\PY{l+m+mf}{0.5}\PY{p}{,} \PY{l+m+mf}{0.5}\PY{o}{*}\PY{n}{np}\PY{o}{.}\PY{n}{sqrt}\PY{p}{(}\PY{l+m+mi}{3}\PY{p}{)}\PY{p}{)}\PY{p}{,} \PY{o}{.}\PY{l+m+mi}{5}\PY{p}{,} \PY{n}{color}\PY{o}{=}\PY{l+s+s1}{\PYZsq{}}\PY{l+s+s1}{b}\PY{l+s+s1}{\PYZsq{}}\PY{p}{)}
\PY{n}{circle4} \PY{o}{=} \PY{n}{plt}\PY{o}{.}\PY{n}{Circle}\PY{p}{(}\PY{p}{(}\PY{l+m+mi}{1}\PY{p}{,} \PY{l+m+mi}{0}\PY{p}{)}\PY{p}{,} \PY{o}{.}\PY{l+m+mi}{5}\PY{p}{,} \PY{n}{color}\PY{o}{=}\PY{l+s+s1}{\PYZsq{}}\PY{l+s+s1}{magenta}\PY{l+s+s1}{\PYZsq{}}\PY{p}{)}
\PY{n}{circle5} \PY{o}{=} \PY{n}{plt}\PY{o}{.}\PY{n}{Circle}\PY{p}{(}\PY{p}{(}\PY{o}{\PYZhy{}}\PY{l+m+mf}{0.5}\PY{p}{,} \PY{o}{\PYZhy{}}\PY{l+m+mf}{0.5}\PY{o}{*}\PY{n}{np}\PY{o}{.}\PY{n}{sqrt}\PY{p}{(}\PY{l+m+mi}{3}\PY{p}{)}\PY{p}{)}\PY{p}{,} \PY{o}{.}\PY{l+m+mi}{5}\PY{p}{,} \PY{n}{color}\PY{o}{=}\PY{l+s+s1}{\PYZsq{}}\PY{l+s+s1}{g}\PY{l+s+s1}{\PYZsq{}}\PY{p}{)}
\PY{n}{circle6} \PY{o}{=} \PY{n}{plt}\PY{o}{.}\PY{n}{Circle}\PY{p}{(}\PY{p}{(}\PY{l+m+mf}{0.5}\PY{p}{,} \PY{o}{\PYZhy{}}\PY{l+m+mf}{0.5}\PY{o}{*}\PY{n}{np}\PY{o}{.}\PY{n}{sqrt}\PY{p}{(}\PY{l+m+mi}{3}\PY{p}{)}\PY{p}{)}\PY{p}{,} \PY{o}{.}\PY{l+m+mi}{5}\PY{p}{,} \PY{n}{color}\PY{o}{=}\PY{l+s+s1}{\PYZsq{}}\PY{l+s+s1}{b}\PY{l+s+s1}{\PYZsq{}}\PY{p}{)}
\PY{n}{circle7} \PY{o}{=} \PY{n}{plt}\PY{o}{.}\PY{n}{Circle}\PY{p}{(}\PY{p}{(}\PY{o}{\PYZhy{}}\PY{l+m+mi}{1}\PY{p}{,} \PY{l+m+mi}{0}\PY{p}{)}\PY{p}{,} \PY{o}{.}\PY{l+m+mi}{5}\PY{p}{,} \PY{n}{color}\PY{o}{=}\PY{l+s+s1}{\PYZsq{}}\PY{l+s+s1}{k}\PY{l+s+s1}{\PYZsq{}}\PY{p}{)}
\PY{c+c1}{\PYZsh{} change default range so that new circles will work}
\PY{n}{ax}\PY{o}{.}\PY{n}{set\PYZus{}xlim}\PY{p}{(}\PY{p}{(}\PY{o}{\PYZhy{}}\PY{l+m+mi}{4}\PY{p}{,} \PY{l+m+mi}{4}\PY{p}{)}\PY{p}{)}
\PY{n}{ax}\PY{o}{.}\PY{n}{set\PYZus{}ylim}\PY{p}{(}\PY{p}{(}\PY{o}{\PYZhy{}}\PY{l+m+mi}{4}\PY{p}{,} \PY{l+m+mi}{4}\PY{p}{)}\PY{p}{)}
\PY{n}{ax}\PY{o}{.}\PY{n}{add\PYZus{}patch}\PY{p}{(}\PY{n}{circle1}\PY{p}{)}
\PY{n}{ax}\PY{o}{.}\PY{n}{add\PYZus{}patch}\PY{p}{(}\PY{n}{circle2}\PY{p}{)}
\PY{n}{ax}\PY{o}{.}\PY{n}{add\PYZus{}patch}\PY{p}{(}\PY{n}{circle3}\PY{p}{)}
\PY{n}{ax}\PY{o}{.}\PY{n}{add\PYZus{}patch}\PY{p}{(}\PY{n}{circle4}\PY{p}{)}
\PY{n}{ax}\PY{o}{.}\PY{n}{add\PYZus{}patch}\PY{p}{(}\PY{n}{circle5}\PY{p}{)}
\PY{n}{ax}\PY{o}{.}\PY{n}{add\PYZus{}patch}\PY{p}{(}\PY{n}{circle6}\PY{p}{)}
\PY{n}{ax}\PY{o}{.}\PY{n}{add\PYZus{}patch}\PY{p}{(}\PY{n}{circle7}\PY{p}{)}
\PY{n}{plt}\PY{o}{.}\PY{n}{title}\PY{p}{(}\PY{l+s+s2}{\PYZdq{}}\PY{l+s+s2}{Hexagonal Grid}\PY{l+s+s2}{\PYZdq{}}\PY{p}{,}\PY{n}{fontsize}\PY{o}{=}\PY{l+m+mi}{20}\PY{p}{)}
\PY{n}{plt}\PY{o}{.}\PY{n}{xlabel}\PY{p}{(}\PY{l+s+sa}{r}\PY{l+s+s1}{\PYZsq{}}\PY{l+s+s1}{\PYZdl{}f\PYZus{}1\PYZdl{}}\PY{l+s+s1}{\PYZsq{}}\PY{p}{,}\PY{n}{fontsize}\PY{o}{=}\PY{l+m+mi}{20}\PY{p}{)}
\PY{n}{plt}\PY{o}{.}\PY{n}{ylabel}\PY{p}{(}\PY{l+s+sa}{r}\PY{l+s+s1}{\PYZsq{}}\PY{l+s+s1}{\PYZdl{}f\PYZus{}2\PYZdl{}}\PY{l+s+s1}{\PYZsq{}}\PY{p}{,}\PY{n}{fontsize}\PY{o}{=}\PY{l+m+mi}{20}\PY{p}{)}
\end{Verbatim}
\end{tcolorbox}

            \begin{tcolorbox}[breakable, size=fbox, boxrule=.5pt, pad at break*=1mm, opacityfill=0]
\prompt{Out}{outcolor}{25}{\boxspacing}
\begin{Verbatim}[commandchars=\\\{\}]
Text(0, 0.5, '\$f\_2\$')
\end{Verbatim}
\end{tcolorbox}
        
    \begin{center}
    \adjustimage{max size={0.9\linewidth}{0.9\paperheight}}{HW_1_files/HW_1_30_1.png}
    \end{center}
    { \hspace*{\fill} \\}
    
    To recover the signal exactly from the frequency spectrum, we should
conduct sampling avoiding aliasing. The minimum sampling frequency for
this in regular grid turned out to be 1 \(cycles/meter\), that is
maximum 1 meter period.

    \hypertarget{q3a}{%
\section{Q3a}\label{q3a}}

    \begin{tcolorbox}[breakable, size=fbox, boxrule=1pt, pad at break*=1mm,colback=cellbackground, colframe=cellborder]
\prompt{In}{incolor}{26}{\boxspacing}
\begin{Verbatim}[commandchars=\\\{\}]
\PY{k+kn}{from} \PY{n+nn}{skimage}\PY{n+nn}{.}\PY{n+nn}{color} \PY{k+kn}{import} \PY{n}{rgb2gray}
\PY{n}{image} \PY{o}{=} \PY{n}{rgb2gray}\PY{p}{(}\PY{n}{plt}\PY{o}{.}\PY{n}{imread}\PY{p}{(}\PY{l+s+s2}{\PYZdq{}}\PY{l+s+s2}{256by256grayscaleLena.png}\PY{l+s+s2}{\PYZdq{}}\PY{p}{)}\PY{p}{)}
\PY{n}{minsize} \PY{o}{=} \PY{l+m+mi}{0}
\PY{n}{maxsize} \PY{o}{=} \PY{n}{image}\PY{o}{.}\PY{n}{shape}\PY{p}{[}\PY{l+m+mi}{1}\PY{p}{]}
\PY{n}{X} \PY{o}{=} \PY{n}{np}\PY{o}{.}\PY{n}{arange}\PY{p}{(}\PY{n}{minsize}\PY{p}{,} \PY{n}{maxsize}\PY{p}{,} \PY{l+m+mi}{1}\PY{p}{)}
\PY{n}{Y} \PY{o}{=} \PY{n}{np}\PY{o}{.}\PY{n}{arange}\PY{p}{(}\PY{n}{minsize}\PY{p}{,} \PY{n}{maxsize}\PY{p}{,} \PY{l+m+mi}{1}\PY{p}{)}
\PY{n}{X}\PY{p}{,} \PY{n}{Y} \PY{o}{=} \PY{n}{np}\PY{o}{.}\PY{n}{meshgrid}\PY{p}{(}\PY{n}{X}\PY{p}{,} \PY{n}{Y}\PY{p}{)}
\PY{n}{plt}\PY{o}{.}\PY{n}{imshow}\PY{p}{(}\PY{n}{image}\PY{p}{,}\PY{n}{cmap}\PY{o}{=}\PY{l+s+s2}{\PYZdq{}}\PY{l+s+s2}{gray}\PY{l+s+s2}{\PYZdq{}}\PY{p}{)}
\end{Verbatim}
\end{tcolorbox}

            \begin{tcolorbox}[breakable, size=fbox, boxrule=.5pt, pad at break*=1mm, opacityfill=0]
\prompt{Out}{outcolor}{26}{\boxspacing}
\begin{Verbatim}[commandchars=\\\{\}]
<matplotlib.image.AxesImage at 0x25fe829bbb0>
\end{Verbatim}
\end{tcolorbox}
        
    \begin{center}
    \adjustimage{max size={0.9\linewidth}{0.9\paperheight}}{HW_1_files/HW_1_33_1.png}
    \end{center}
    { \hspace*{\fill} \\}
    
    \begin{tcolorbox}[breakable, size=fbox, boxrule=1pt, pad at break*=1mm,colback=cellbackground, colframe=cellborder]
\prompt{In}{incolor}{27}{\boxspacing}
\begin{Verbatim}[commandchars=\\\{\}]
\PY{c+c1}{\PYZsh{} Plot the surface.}
\PY{n}{fig} \PY{o}{=} \PY{n}{plt}\PY{o}{.}\PY{n}{figure}\PY{p}{(}\PY{n}{figsize}\PY{o}{=}\PY{n}{plt}\PY{o}{.}\PY{n}{figaspect}\PY{p}{(}\PY{o}{.}\PY{l+m+mi}{5}\PY{p}{)}\PY{p}{)}
\PY{n}{ax} \PY{o}{=} \PY{n}{fig}\PY{o}{.}\PY{n}{add\PYZus{}subplot}\PY{p}{(}\PY{l+m+mi}{1}\PY{p}{,} \PY{l+m+mi}{2}\PY{p}{,} \PY{l+m+mi}{1}\PY{p}{,} \PY{n}{projection}\PY{o}{=}\PY{l+s+s1}{\PYZsq{}}\PY{l+s+s1}{3d}\PY{l+s+s1}{\PYZsq{}}\PY{p}{)}
\PY{n}{surf} \PY{o}{=} \PY{n}{ax}\PY{o}{.}\PY{n}{plot\PYZus{}surface}\PY{p}{(}\PY{n}{X}\PY{p}{,} \PY{n}{Y}\PY{p}{,} \PY{l+m+mi}{10}\PY{o}{*}\PY{n}{np}\PY{o}{.}\PY{n}{log10}\PY{p}{(}\PY{n}{np}\PY{o}{.}\PY{n}{abs}\PY{p}{(}\PY{n}{fft2}\PY{p}{(}\PY{n}{image}\PY{p}{)}\PY{p}{)}\PY{p}{)}\PY{p}{,}\PY{n}{rstride}\PY{o}{=}\PY{l+m+mi}{1}\PY{p}{,} 
    \PY{n}{cstride}\PY{o}{=}\PY{l+m+mi}{1}\PY{p}{,} \PY{n}{cmap}\PY{o}{=}\PY{n}{cm}\PY{o}{.}\PY{n}{magma}\PY{p}{,} \PY{n}{linewidth}\PY{o}{=}\PY{l+m+mi}{0}\PY{p}{,} \PY{n}{antialiased}\PY{o}{=}\PY{k+kc}{False}\PY{p}{)}
\PY{n}{ax}\PY{o}{.}\PY{n}{set\PYZus{}zlim}\PY{p}{(}\PY{l+m+mi}{0}\PY{p}{,}\PY{l+m+mi}{40}\PY{p}{)}
\PY{n}{ax}\PY{o}{.}\PY{n}{view\PYZus{}init}\PY{p}{(}\PY{l+m+mi}{30}\PY{p}{,} \PY{l+m+mi}{30}\PY{p}{)}
\PY{n}{ax}\PY{o}{.}\PY{n}{set\PYZus{}xticks}\PY{p}{(}\PY{n}{np}\PY{o}{.}\PY{n}{linspace}\PY{p}{(}\PY{o}{\PYZhy{}}\PY{l+m+mi}{128}\PY{p}{,}\PY{l+m+mi}{128}\PY{p}{,}\PY{l+m+mi}{5}\PY{p}{)}\PY{p}{)}
\PY{n}{ax}\PY{o}{.}\PY{n}{set\PYZus{}yticks}\PY{p}{(}\PY{n}{np}\PY{o}{.}\PY{n}{linspace}\PY{p}{(}\PY{o}{\PYZhy{}}\PY{l+m+mi}{128}\PY{p}{,}\PY{l+m+mi}{128}\PY{p}{,}\PY{l+m+mi}{5}\PY{p}{)}\PY{p}{)}
\PY{n}{plt}\PY{o}{.}\PY{n}{title}\PY{p}{(}\PY{l+s+s2}{\PYZdq{}}\PY{l+s+s2}{Non\PYZhy{}centered DFT}\PY{l+s+s2}{\PYZdq{}}\PY{p}{)}

\PY{n}{ax} \PY{o}{=} \PY{n}{fig}\PY{o}{.}\PY{n}{add\PYZus{}subplot}\PY{p}{(}\PY{l+m+mi}{1}\PY{p}{,} \PY{l+m+mi}{2}\PY{p}{,} \PY{l+m+mi}{2}\PY{p}{,} \PY{n}{projection}\PY{o}{=}\PY{l+s+s1}{\PYZsq{}}\PY{l+s+s1}{3d}\PY{l+s+s1}{\PYZsq{}}\PY{p}{)}
\PY{n}{surf} \PY{o}{=} \PY{n}{ax}\PY{o}{.}\PY{n}{plot\PYZus{}surface}\PY{p}{(}\PY{n}{X}\PY{o}{\PYZhy{}}\PY{l+m+mi}{128}\PY{p}{,} \PY{n}{Y}\PY{o}{\PYZhy{}}\PY{l+m+mi}{128}\PY{p}{,} \PY{l+m+mi}{10}\PY{o}{*}\PY{n}{np}\PY{o}{.}\PY{n}{log10}\PY{p}{(}\PY{n}{np}\PY{o}{.}\PY{n}{abs}\PY{p}{(}\PY{n}{fftshift}\PY{p}{(}\PY{n}{fft2}\PY{p}{(}\PY{n}{image}\PY{p}{)}\PY{p}{)}\PY{p}{)}\PY{p}{)}\PY{p}{,}\PY{n}{rstride}\PY{o}{=}\PY{l+m+mi}{1}\PY{p}{,} 
    \PY{n}{cstride}\PY{o}{=}\PY{l+m+mi}{1}\PY{p}{,} \PY{n}{cmap}\PY{o}{=}\PY{n}{cm}\PY{o}{.}\PY{n}{magma}\PY{p}{,} \PY{n}{linewidth}\PY{o}{=}\PY{l+m+mi}{0}\PY{p}{,} \PY{n}{antialiased}\PY{o}{=}\PY{k+kc}{False}\PY{p}{)}
\PY{n}{ax}\PY{o}{.}\PY{n}{set\PYZus{}zlim}\PY{p}{(}\PY{l+m+mi}{0}\PY{p}{,}\PY{l+m+mi}{40}\PY{p}{)}
\PY{n}{ax}\PY{o}{.}\PY{n}{view\PYZus{}init}\PY{p}{(}\PY{l+m+mi}{30}\PY{p}{,} \PY{l+m+mi}{30}\PY{p}{)}
\PY{n}{ax}\PY{o}{.}\PY{n}{set\PYZus{}xticks}\PY{p}{(}\PY{n}{np}\PY{o}{.}\PY{n}{linspace}\PY{p}{(}\PY{o}{\PYZhy{}}\PY{l+m+mi}{128}\PY{p}{,}\PY{l+m+mi}{128}\PY{p}{,}\PY{l+m+mi}{5}\PY{p}{)}\PY{p}{)}
\PY{n}{ax}\PY{o}{.}\PY{n}{set\PYZus{}yticks}\PY{p}{(}\PY{n}{np}\PY{o}{.}\PY{n}{linspace}\PY{p}{(}\PY{o}{\PYZhy{}}\PY{l+m+mi}{128}\PY{p}{,}\PY{l+m+mi}{128}\PY{p}{,}\PY{l+m+mi}{5}\PY{p}{)}\PY{p}{)}
\PY{n}{plt}\PY{o}{.}\PY{n}{title}\PY{p}{(}\PY{l+s+s2}{\PYZdq{}}\PY{l+s+s2}{Centered DFT}\PY{l+s+s2}{\PYZdq{}}\PY{p}{)}
\end{Verbatim}
\end{tcolorbox}

            \begin{tcolorbox}[breakable, size=fbox, boxrule=.5pt, pad at break*=1mm, opacityfill=0]
\prompt{Out}{outcolor}{27}{\boxspacing}
\begin{Verbatim}[commandchars=\\\{\}]
Text(0.5, 0.92, 'Centered DFT')
\end{Verbatim}
\end{tcolorbox}
        
    \begin{center}
    \adjustimage{max size={0.9\linewidth}{0.9\paperheight}}{HW_1_files/HW_1_34_1.png}
    \end{center}
    { \hspace*{\fill} \\}
    
    \hypertarget{q3b}{%
\section{Q3b}\label{q3b}}

    Take the image and upsample with zeroes in every 2 elements.

    \begin{tcolorbox}[breakable, size=fbox, boxrule=1pt, pad at break*=1mm,colback=cellbackground, colframe=cellborder]
\prompt{In}{incolor}{28}{\boxspacing}
\begin{Verbatim}[commandchars=\\\{\}]
\PY{n}{image} \PY{o}{=} \PY{n}{rgb2gray}\PY{p}{(}\PY{n}{plt}\PY{o}{.}\PY{n}{imread}\PY{p}{(}\PY{l+s+s2}{\PYZdq{}}\PY{l+s+s2}{256by256grayscaleLena.png}\PY{l+s+s2}{\PYZdq{}}\PY{p}{)}\PY{p}{)}
\PY{n}{cropped\PYZus{}image} \PY{o}{=} \PY{n}{image}\PY{p}{[}\PY{l+m+mi}{60}\PY{p}{:}\PY{l+m+mi}{188}\PY{p}{,}\PY{l+m+mi}{60}\PY{p}{:}\PY{l+m+mi}{188}\PY{p}{]}
\PY{n}{plt}\PY{o}{.}\PY{n}{imshow}\PY{p}{(}\PY{n}{cropped\PYZus{}image}\PY{p}{)}
\end{Verbatim}
\end{tcolorbox}

            \begin{tcolorbox}[breakable, size=fbox, boxrule=.5pt, pad at break*=1mm, opacityfill=0]
\prompt{Out}{outcolor}{28}{\boxspacing}
\begin{Verbatim}[commandchars=\\\{\}]
<matplotlib.image.AxesImage at 0x25fef01aa90>
\end{Verbatim}
\end{tcolorbox}
        
    \begin{center}
    \adjustimage{max size={0.9\linewidth}{0.9\paperheight}}{HW_1_files/HW_1_37_1.png}
    \end{center}
    { \hspace*{\fill} \\}
    
    Calculate its 2D Fourier transform.

    \begin{tcolorbox}[breakable, size=fbox, boxrule=1pt, pad at break*=1mm,colback=cellbackground, colframe=cellborder]
\prompt{In}{incolor}{29}{\boxspacing}
\begin{Verbatim}[commandchars=\\\{\}]
\PY{n}{s1}\PY{p}{,}\PY{n}{s2} \PY{o}{=} \PY{n}{cropped\PYZus{}image}\PY{o}{.}\PY{n}{shape}
\PY{n}{upsampled\PYZus{}image} \PY{o}{=} \PY{n}{np}\PY{o}{.}\PY{n}{zeros}\PY{p}{(}\PY{p}{(}\PY{l+m+mi}{2}\PY{o}{*}\PY{n}{s1}\PY{p}{,}\PY{l+m+mi}{2}\PY{o}{*}\PY{n}{s2}\PY{p}{)}\PY{p}{)}
\PY{n}{upsampled\PYZus{}image}\PY{p}{[}\PY{p}{:}\PY{p}{:}\PY{l+m+mi}{2}\PY{p}{,}\PY{p}{:}\PY{p}{:}\PY{l+m+mi}{2}\PY{p}{]} \PY{o}{=} \PY{n}{cropped\PYZus{}image}
\PY{n}{FFT\PYZus{}upsampled} \PY{o}{=} \PY{n}{fft2}\PY{p}{(}\PY{n}{upsampled\PYZus{}image}\PY{p}{)}
\PY{n}{plt}\PY{o}{.}\PY{n}{imshow}\PY{p}{(}\PY{n}{np}\PY{o}{.}\PY{n}{log10}\PY{p}{(}\PY{n}{np}\PY{o}{.}\PY{n}{abs}\PY{p}{(}\PY{n}{FFT\PYZus{}upsampled}\PY{p}{)}\PY{p}{)}\PY{p}{)}
\end{Verbatim}
\end{tcolorbox}

            \begin{tcolorbox}[breakable, size=fbox, boxrule=.5pt, pad at break*=1mm, opacityfill=0]
\prompt{Out}{outcolor}{29}{\boxspacing}
\begin{Verbatim}[commandchars=\\\{\}]
<matplotlib.image.AxesImage at 0x25fe8cd2700>
\end{Verbatim}
\end{tcolorbox}
        
    \begin{center}
    \adjustimage{max size={0.9\linewidth}{0.9\paperheight}}{HW_1_files/HW_1_39_1.png}
    \end{center}
    { \hspace*{\fill} \\}
    
    Filter the ``central'' portion of the frequency spectrum out with an
ideal low pass filter.

    \begin{tcolorbox}[breakable, size=fbox, boxrule=1pt, pad at break*=1mm,colback=cellbackground, colframe=cellborder]
\prompt{In}{incolor}{30}{\boxspacing}
\begin{Verbatim}[commandchars=\\\{\}]
\PY{c+c1}{\PYZsh{}\PYZsh{} Take the center portion, equate remaining to 0 }
\PY{n}{shifted} \PY{o}{=} \PY{n}{fftshift}\PY{p}{(}\PY{n}{FFT\PYZus{}upsampled}\PY{p}{)}
\PY{n}{Filtered\PYZus{}FFT} \PY{o}{=} \PY{n}{np}\PY{o}{.}\PY{n}{zeros\PYZus{}like}\PY{p}{(}\PY{n}{FFT\PYZus{}upsampled}\PY{p}{)}
\PY{n}{Filtered\PYZus{}FFT}\PY{p}{[}\PY{l+m+mi}{64}\PY{p}{:}\PY{l+m+mi}{192}\PY{p}{,}\PY{l+m+mi}{64}\PY{p}{:}\PY{l+m+mi}{192}\PY{p}{]} \PY{o}{=} \PY{n}{FFT\PYZus{}upsampled}\PY{p}{[}\PY{l+m+mi}{64}\PY{p}{:}\PY{l+m+mi}{192}\PY{p}{,}\PY{l+m+mi}{64}\PY{p}{:}\PY{l+m+mi}{192}\PY{p}{]}
\PY{n}{Filtered\PYZus{}FFT} \PY{o}{=} \PY{n}{fftshift}\PY{p}{(}\PY{n}{Filtered\PYZus{}FFT}\PY{p}{)}
\PY{n}{plt}\PY{o}{.}\PY{n}{imshow}\PY{p}{(}\PY{n}{np}\PY{o}{.}\PY{n}{log10}\PY{p}{(}\PY{n}{np}\PY{o}{.}\PY{n}{abs}\PY{p}{(}\PY{n}{Filtered\PYZus{}FFT}\PY{p}{)}\PY{p}{)}\PY{p}{)}
\end{Verbatim}
\end{tcolorbox}

    \begin{Verbatim}[commandchars=\\\{\}]
<ipython-input-30-83fcb38667d6>:6: RuntimeWarning: divide by zero encountered in
log10
  plt.imshow(np.log10(np.abs(Filtered\_FFT)))
    \end{Verbatim}

            \begin{tcolorbox}[breakable, size=fbox, boxrule=.5pt, pad at break*=1mm, opacityfill=0]
\prompt{Out}{outcolor}{30}{\boxspacing}
\begin{Verbatim}[commandchars=\\\{\}]
<matplotlib.image.AxesImage at 0x25fe87df7c0>
\end{Verbatim}
\end{tcolorbox}
        
    \begin{center}
    \adjustimage{max size={0.9\linewidth}{0.9\paperheight}}{HW_1_files/HW_1_41_2.png}
    \end{center}
    { \hspace*{\fill} \\}
    
    Take the inverse 2D FFT to obtain the interpolated image.

    \begin{tcolorbox}[breakable, size=fbox, boxrule=1pt, pad at break*=1mm,colback=cellbackground, colframe=cellborder]
\prompt{In}{incolor}{31}{\boxspacing}
\begin{Verbatim}[commandchars=\\\{\}]
\PY{n}{interpolated\PYZus{}image} \PY{o}{=} \PY{n}{np}\PY{o}{.}\PY{n}{real}\PY{p}{(}\PY{n}{ifft2}\PY{p}{(}\PY{n}{Filtered\PYZus{}FFT}\PY{p}{)}\PY{p}{)}
\PY{n}{plt}\PY{o}{.}\PY{n}{imshow}\PY{p}{(}\PY{n}{interpolated\PYZus{}image}\PY{p}{)}
\end{Verbatim}
\end{tcolorbox}

            \begin{tcolorbox}[breakable, size=fbox, boxrule=.5pt, pad at break*=1mm, opacityfill=0]
\prompt{Out}{outcolor}{31}{\boxspacing}
\begin{Verbatim}[commandchars=\\\{\}]
<matplotlib.image.AxesImage at 0x25ff03b0f10>
\end{Verbatim}
\end{tcolorbox}
        
    \begin{center}
    \adjustimage{max size={0.9\linewidth}{0.9\paperheight}}{HW_1_files/HW_1_43_1.png}
    \end{center}
    { \hspace*{\fill} \\}
    

    % Add a bibliography block to the postdoc
    
    
    
\end{document}
